\chapter{Mechanisms Controlling the Land/Sea Contrast} 

\label{mechanisms} 

\lhead{Chapter 4. \emph{Mechanisms}} 

\section{Introduction}
\paragraph{Discuss GW theories, relevance to interannual var.}

%----------------------------------------------------------------------------------------
%	influence of SSTs on Continental temps
%----------------------------------------------------------------------------------------

\section{The Influence of Tropical Oceans on Land}

\paragraph{Strong land/ocean connection in tropics}

\subsection{1K Experiments}

Figure \ref{fig:1K} shows the surface temperature response (control removed) 
from the 1K experiments; where 1K was added to the oceans in the tropics, 
extra-tropics or globally. In response to a tropical SST perturbation there is a 
large tropical response, greatest over equatorial South America and Africa, 
India and the maritime continent (Fig.\ref{fig:1K} a). The tropical 1K ocean 
perturbation leads to $T_{land}>+1$K in most tropical areas. Thus the SST 
forcing is amplified. The extra-tropical land the response to the tropical 
forcing is not significant everywhere, but some regions also show an amplified 
response to the tropical SST forcing (e.g. central Asia and parts of Europe and 
North America).  An extra-tropical $T_{land}$ response to tropical SST is seen 
for seasonal averages in the winter months of each hemisphere, the Northen 
Hemsiphere winter response is shown in Figure \ref{fig:1K}e.

When looking at the annual mean response of $T_{land}$ to extra-tropical SST 
perturbations there is little significant response, however for seasonal 
averages both hemispheres show a significant response in their respective winter 
months, shown for the Northern Hemisphere in Figure \ref{fig:1K} e-g. The 
response of $T_{land}$ is also amplified in some regions relative to the initial 
perturbation.  However, the extra-tropical forcing again leads to a weaker land 
response than the tropical forcing, as was also found in the AMIP simulations.  
Also similar to the AMIP simulations we again find that the  global SST forcing 
has a bigger impact than the tropical only forcing for the annual mean. In 
addition the response of the global SST forcing is greater than the 
superposition of the tropical and extra-tropical forcing (comparing 
Fig.\ref{fig:1K} c and d).  This again indirectly suggests that the 
extra-tropical SST forcing does lead to a significant land response.


\subsection{Tropical Troposphere}
\begin{itemize}
	\item Strong land/ocean connection
	\item Variability in SSTs lead to amplified temperature variation in 
		troposphere.
	\item Tropical tropospheric unable to sustain large temperature gradients.
	\item Perturbations efficiently transported around tropical regions.
	\item Uniform perturbations over land and ocean.
	\item In the mean - response of surface temperature differs more over land 
		and ocean.
	\item Increase in variability due to ocean forcing.
\end{itemize}

\subsection{Half-AMIP Experiments}

In Figure \ref{fig:ftest} f-tests are used to measure the increase in annual 
temperature variability due to SST variability at the surface and at the 300hPa 
pressure level relative to a simulation with fixed SST climatology. Figure  
\ref{fig:ftest} a) and b) show that global SST variability has a substantial 
impact on the tropical atmospheric and surface temperature variability. However, 
in the extra-tropical regions the impact is much weaker, but still statistically 
significant in some regions.

In order to separate the influence of the tropical SST variability from that of 
the extra-tropical SST, we repeat the AMIP experiment forced with the historical 
SST variability just in the tropics or just in the extra-tropical regions. The 
impact of the tropical SST variability is similar to the global SST variability, 
with a clear and strong impact in the tropical regions.  The AMIP simulation 
with just the extra-tropical SST variability has only a very weak to no impact 
on the regional (grid-box scale) atmospheric and surface temperature 
variability.  However, if we compare the global AMIP versus the tropical only 
AMIP run we still can see a somewhat larger increase in variance over land in 
the global AMIP run. This indirectly suggests that the extra-tropical SST 
forcing does play a role, although it is much smaller than the tropical forcing.  
In summary the AMIP experiments suggest a clear tropical SST forcing to the 
atmospheric and land surface temperatures, but a much weaker or no forcing from 
the extra-tropical SST.

%----------------------------------------------------------------------------------------
%	LAPSE RATE
%----------------------------------------------------------------------------------------

\section{The Lapse Rate Mechanism}
\begin{itemize}
	\item Forcing regions over tropical oceans show large amplification in upper 
		troposphere due to moist convection.
	\item Responding tropical areas - both land and over ocean - similar, i.e.  
		regresion profile figure.
	\item Continental lapse rate more variable due to lower mean humidity.
\end{itemize}


\subsection{The Weak Temperature Gradient Approximation}
\begin{itemize}
	\item What is WTG.
	\item Approximation of tropical dynamics
	\item Useful for understanding convection, roll of moisture.
	\item Application to land/ocean contrast.
	\item ...
\end{itemize}

\subsection{Experiments with a Single-Column Model}
\paragraph{explain setup}
\begin{itemize}
	\item WTG assumption
	\item Generation of atmospheric profiles
	\item Parameter values;
		\begin{itemize}
			\item Land surface - Evaporative fraction
			\item WTG - pressure level of fixed temp profile
			\item Interactive clouds
		\end{itemize}
\end{itemize}
\paragraph{Values of l/s contrast for varying land/tropos parameters}
\paragraph{Atmospheric profiles}
\paragraph{Variability and instability - Role of convection}


\subsection{Experiments with a Two-Dimensional Model}

\paragraph{explain setup}
\begin{itemize}
	\item Land/Ocean column configurations
	\item Parameter values;
		\begin{itemize}
			\item Land surface - Evaporative fraction
			\item Interactive clouds
		\end{itemize}
\end{itemize}

\paragraph{Values of l/s contrast for varying land parameters}
\paragraph{Atmospheric profiles in temp, RH, etc}
\paragraph{Variability and instability - Role of convection and clouds}
\paragraph{Circulation response}

\subsection{Comparison of Simplified Models and GCM results}

\paragraph{Generation of suitable l/s contrast at varying latitudes/evap fraction}
\paragraph{Mapping results}
\paragraph{Differences/Similarities between GCM/RCM}


%----------------------------------------------------------------------------------------
%	ENSO
%----------------------------------------------------------------------------------------

\section{ENSO}
\begin{itemize}
	\item ENSO is source of tropical inter-annual variability.
	\item Temperature anomaly seen throughout tropical troposphere
	\item Land responds quickly - within a month - to initial anomaly.
	\item Remote tropical oceans response is delayed by 3-4 months.
	\item Timing of delay of remote tropical oceans is a function of mixed layer 
		depth.
	\item Land responds to the delayed forcing by remote trop oceans.
	\item Mean land response appears to be delayed, i.e. when plotted with lag 
		correlations.
	\item Delayed land response is linear combination of initial response to 
		Pacific ocean and response to delayed remote tropical oceans.
	\item Land response is amplified due to previously discussed process.
	\item The land response is further amplified by secondary response to remote 
		tropics.
\end{itemize}

\subsection{Pacemaker Experiments}


On inter-annual timescales ENSO is the most significant global climate driver. 
It is therefore remarkable that in the analysis of the CMIP5 model simulations 
the NINO3 region did not show up with a high correlation to global $T_{land}$ 
(see Fig.\ref{fig:cmip_cormap}). The ENSO region in the tropical Pacific has a 
lower correlation with $T_{land}$ than adjacent regions and the other ocean 
basins. Using the combined monthly mean CMIP5 surface temperature anomalies, 
Fig.\ref{fig:xcor} e) shows the lagged correlations between NINO3 SST and global 
$T_{land}$. The NINO3 region is seen to lead global land by 4 months.  Typically 
land has a fast response time to forcings, which would not result in a 
4 month delay, so this result suggests that the full land response is not 
directly forced by the NINO3 SST but is most likely caused by something else. 
This other forcing may be delayed to the ENSO variability by about 4 months. 
Since we have seen in Fig.\ref{fig:cmip_cormap} that global $T_{land}$ is highly 
correlated to other tropical ocean SST, it seems likely that $T_{land}$ is 
linked to the slower ocean response in the remote tropical oceans and not 
directly to the NINO3 region.

To address this question we conducted a series of idealised ENSO-response 
experiments. In the first experiment we prescribe an oscillating ENSO pattern (a 
regression between NINO3 and SSTs shown in Fig.\ref{fig:regpat}) in the tropical 
Pacific and fixed SST climatologies elsewhere.  The oscillation period of the 
ENSO signal is 4 years, peaking in January. In the second experiment we allow 
SST variability outside the tropical Pacific simulated by a simple slab ocean 
model.  Thus, in the second experiment the global ocean SSTs can respond to the 
oscillating ENSO pattern forcing.

Figure \ref{fig:xcor} (i-l) shows cross-correlations from the ENSO-FIXSST and 
ENSO-Slab forcing experiments. In i) and j) we see that for the fixed SST 
experiment the global and tropical land responds to the ENSO-like forcing (red 
line), and does so without the delay seen in Figure \ref{fig:xcor} e). When a 
slab ocean is introduced the land responds with a realistic delay of around 4 
months.  The peak slab ocean response is at 6 months, implying that the land is 
responding immediately to the initial Pacific ocean forcing and then 
subsequently to the delayed slab ocean response. The delayed land response is 
also associated with a higher correlation to the NINO3 region. Comparing the 
global and tropical averages, the main difference is the magnitude of the peak 
correlation, but in the tropics the slab ocean also results in the peak land 
correlation being higher than the peak ocean correlation. So the delayed 
response of the remote tropical oceans to a Pacific ocean forcing explains both 
the delayed land response and some part of the amplification of the oceanic 
temperature signal over land. In the extra-tropics there is only a very weak 
influence of the ENSO forcing on land temperatures in the sensitivity 
experiments (Fig.  \ref{fig:xcor} g, h), and the tropical Pacific has little 
influence on the slab ocean in the extra-tropics. The observations and CMIP5 
models also don't show a significant relationship between the extra-tropics and 
NINO3.

%----------------------------------------------------------------------------------------
%	Extra-Tropics
%----------------------------------------------------------------------------------------

\section{Extra-Tropical Response}
\begin{itemize}
	\item Relationship between land and ocean very different in extra-tropics.
	\item Discuss Barsugli, Battitsi atmospheric forcing of SST anomalies.
	\item Largest discrepancies between models and observations exist in 
		extra-tropics.
	\item Changes in cloud cover, winds induced by tropical SSTs can impact 
		extra-tropics, i.e. pacemaker and GREB model results.
	\item COWL - COld Ocean, Warm Land theory...
\end{itemize}

%----------------------------------------------------------------------------------------
%	Regional Response / GREB
%----------------------------------------------------------------------------------------

\section{Regional Response}

\begin{itemize}
	\item Lapse rate theory works well in the mean continental response over 
		land.
	\item According to theory it would be expected that dryer regions would have 
		larger response, since difference in humidity is primary driver of the 
		increased variability over land.
	\item We see from regional response that the magnitude of variability is not 
		correlated with the dryness of the land surface.
	\item In addition to thermodynamic changes, tropical SSTs induce changes in 
		cloud cover, energy transport by winds, soil moisture changes.
	\item Regional importance of these discussed...
\end{itemize}


\subsection{GREB Model}

\subsection{Deconstructing the GCM response}
\paragraph{motivation}
\paragraph{Discuss variables used/available}
\paragraph{Response of pacemaker exp.}
\paragraph{Response of amip exp.}


\subsection{Results - Pacemaker}
\paragraph{u, v winds}
\paragraph{Cloud cover}
\paragraph{Soil Moisture}
\paragraph{SST}
\paragraph{Combined response}


\subsection{AMIP Experiments}

\paragraph{u, v winds}
\paragraph{Cloud cover}
\paragraph{Soil Moisture}
\paragraph{SST}
\paragraph{Combined response}


