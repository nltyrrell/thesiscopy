\chapter{Mechanisms Controlling the Land/Sea Contrast} 

\label{mechanisms} 

\lhead{Chapter 4. \emph{Mechanisms}} 

\section{Introduction}
\paragraph{Discuss GW theories, relevance to interannual var.}

%----------------------------------------------------------------------------------------
%	Direct influence of SSTs on Continental temps
%----------------------------------------------------------------------------------------

\section{Influence of Tropical Oceans on Land}

%-----------------------------------
%	+1K Experiments
%-----------------------------------
\subsection{+1K experiments}

Figure \ref{fig:1K} shows the surface temperature response (control removed) 
from the +1K experiments; where +1K was added to the oceans in the tropics, 
extra-tropics or globally. In response to a tropical SST perturbation there is a 
large tropical response, greatest over equatorial South America and Africa, 
India and the maritime continent (Fig.\ref{fig:1K} a). The tropical +1K ocean 
perturbation leads to $T_{land}>+1$K in most tropical areas. Thus the SST 
forcing is amplified. The extra-tropical land the response to the tropical 
forcing is not significant everywhere, but some regions also show an amplified 
response to the tropical SST forcing (e.g. central Asia and parts of Europe and 
North America).  An extra-tropical $T_{land}$ response to tropical SST is seen 
for seasonal averages in the winter months of each hemisphere, the Northen 
Hemsiphere winter response is shown in Figure \ref{fig:1K}e.

When looking at the annual mean response of $T_{land}$ to extra-tropical SST 
perturbations there is little significant response, however for seasonal 
averages both hemispheres show a significant response in their respective winter 
months, shown for the Northern Hemisphere in Figure \ref{fig:1K} e-g. The 
response of $T_{land}$ is also amplified in some regions relative to the initial 
perturbation.  However, the extra-tropical forcing again leads to a weaker land 
response than the tropical forcing, as was also found in the AMIP simulations.  
Also similar to the AMIP simulations we again find that the  global SST forcing 
has a bigger impact than the tropical only forcing for the annual mean. In 
addition the response of the global SST forcing is greater than the 
superposition of the tropical and extra-tropical forcing (comparing 
Fig.\ref{fig:1K} c and d).  This again indirectly suggests that the 
extra-tropical SST forcing does lead to a significant land response.

\subsection{Tropical Troposphere}
\begin{itemize}
	\item Strong land/ocean connection
	\item Variability in SSTs lead to amplified temperature variation in 
		troposphere.
	\item Tropical tropospheric unable to sustain large temperature gradients.
	\item Perturbations efficiently transported around tropical regions.
	\item Uniform perturbations over land and ocean.
	\item In the mean - response of surface temperature differs more over land 
		and ocean.
	\item Increase in variability due to ocean forcing.
\end{itemize}

\subsection{Weak Temperature Gradient}
\begin{itemize}
	\item What is WTG.
	\item Approximation of tropical dynamics
	\item Useful for understanding convection, roll of moisture.
	\item Application to land/ocean contrast.
	\item ...
\end{itemize}

\subsection{Lapse Rate}
\begin{itemize}
	\item Forcing regions over tropical oceans show large amplification in upper 
		troposphere due to moist convection.
	\item Responding tropical areas - both land and over ocean - similar, i.e.  
		regresion profile figure.
	\item Continental lapse rate more variable due to lower mean humidity.
\end{itemize}

%----------------------------------------------------------------------------------------
%	ENSO
%----------------------------------------------------------------------------------------

\section{ENSO}
\begin{itemize}
	\item ENSO is source of tropical inter-annual variability.
	\item Temperature anomaly seen throughout tropical troposphere
	\item Land responds quickly - within a month - to initial anomaly.
	\item Remote tropical oceans response is delayed by 3-4 months.
	\item Timing of delay of remote tropical oceans is a function of mixed layer 
		depth.
	\item Land responds to the delayed forcing by remote trop oceans.
	\item Mean land response appears to be delayed, i.e. when plotted with lag 
		correlations.
	\item Delayed land response is linear combination of initial response to 
		Pacific ocean and response to delayed remote tropical oceans.
	\item Land response is amplified due to previously discussed process.
	\item The land response is further amplified by secondary response to remote 
		tropics.
\end{itemize}


%----------------------------------------------------------------------------------------
%	Extra-Tropics
%----------------------------------------------------------------------------------------

\section{Extra-Tropical Response}
\begin{itemize}
	\item Relationship between land and ocean very different in extra-tropics.
	\item Discuss Barsugli, Battitsi atmospheric forcing of SST anomalies.
	\item Largest discrepancies between models and observations exist in 
		extra-tropics.
	\item Changes in cloud cover, winds induced by tropical SSTs can impact 
		extra-tropics, i.e. pacemaker and GREB model results.
	\item COWL - COld Ocean, Warm Land theory...
\end{itemize}

%----------------------------------------------------------------------------------------
%	Regional Response
%----------------------------------------------------------------------------------------

\section{Regional Response}

\begin{itemize}
	\item Lapse rate theory works well in the mean continental response over 
		land.
	\item According to theory it would be expected that dryer regions would have 
		larger response, since difference in humidity is primary driver of the 
		increased variability over land.
	\item We see from regional response that the magnitude of variability is not 
		correlated with the dryness of the land surface.
	\item In addition to thermodynamic changes, tropical SSTs induce changes in 
		cloud cover, energy transport by winds, soil moisture changes.
	\item Regional importance of these discussed...
\end{itemize}