In global warming scenarios, global land surface temperatures ($T_{land}$) warm 
with greater amplitude than sea surface temperatures (SSTs), leading to a 
land/sea warming contrast even in equilibrium. Similarly, the interannual 
variability of $T_{land}$ is larger than the covariant interannual SST
variability, leading to a land/sea contrast in natural variability.  This work 
investigates the land/sea contrast in natural variability based on global 
observations, coupled general circulation model simulations and idealised 
atmospheric general circulation model simulations with different SST forcings. 

The land/sea temperature contrast in interannual variability is found to exist 
in observations and models to a varying extent in global, tropical and 
extra-tropical bands. There is agreement between models and observations in the 
tropics but not the extra-tropics. Causality in the land-sea relationship is 
explored with modelling experiments forced with prescribed SSTs, where an 
amplification of the imposed SST variability is seen over land.  The 
amplification of $T_{land}$ to tropical SST anomalies is due to the enhanced 
upper level atmospheric warming that corresponds with tropical moist convection 
over oceans leading to upper level temperature variations that are larger in 
amplitude than the source SST anomalies. This mechanism is similar to that 
proposed for explaining the equilibrium global warming land/sea warming 
contrast.

The link of the $T_{land}$ to the dominant mode of tropical and global 
interannual climate variability, the El Ni{\~n}o Southern Oscillation (ENSO), is 
found to be an indirect and delayed connection. ENSO SST variability affects the 
oceans outside the tropical Pacific, which in turn leads to a further, amplified 
and delayed response of $T_{land}$.