In global warming scenarios, global land surface temperatures ($T_{land}$) warm 
with greater amplitude than sea surface temperatures (SSTs), leading to a 
land/ocean warming temperature contrast. This land/ocean contrast is not only 
due to the different heat capacities of the land and ocean as it exists for 
transient and equilibrium scenarios.  Similarly, the interannual variability of 
$T_{land}$ is larger than the covariant interannual SST variability, leading to 
a land/ocean temperature contrast in natural variability.  This work 
investigates the land/ocean temperature contrast in natural variability based on 
observations, coupled global model simulations, atmospheric global model 
simulations with different SST forcings, and using idealised models with 
simplified geometry or processes.

The land/ocean temperature contrast in interannual variability is found to exist 
in observations and models to a varying extent in global, tropical and 
extra-tropical bands. There is agreement between models and observations in the 
tropics but not the extra-tropics. Causality in the land-ocean relationship is 
explored with modelling experiments forced with prescribed SSTs, where an 
amplification of the imposed SST variability is seen over land.  The 
amplification of $T_{land}$ to tropical SST anomalies is due to the enhanced 
upper level atmospheric warming that corresponds with tropical moist convection 
over oceans leading to upper level temperature variations that are larger in 
amplitude than the source SST anomalies. This mechanism is similar to that 
proposed for explaining the equilibrium global warming land/ocean warming 
contrast.

The tropospheric structure is studied with single and two column models. It is 
found that realistic values of the land/ocean contrast can be simulated with the 
simplified models, and the atmospheric structure is similar to the mean tropical 
response in global climate models. However on regional scales the simple models 
fail to represent the magnitude or patterns of the global response. The regional 
response is then explored with the Globally Resolved Energy Balance (GREB) 
model.  This model represents the circulation, clouds and soil moisture as 
boundary conditions. Perturbation experiments allow for attribution of the 
regional response of a complex coupled climate model to these boundary 
conditions.

\pagebreak
\textbf{Key Aims:}
\begin{itemize}
	\item Investigate the extent and structure of the interannual land/ocean 
		temperature contrast in observations and models.
	\item Test the importance of ocean forcing on land surface temperature 
		variability.
	\item Explore the mechanisms that cause ocean temperature anomalies to be 
		amplified over land.
	\item Explain regional variations in the continental response to ocean 
		forcing.
\end{itemize}

