\chapter{Introduction} % Main chapter title

\label{introduction} % Change X to a consecutive number; for referencing this 
% chapter elsewhere, use \ref{ChapterX}

\lhead{Chapter 1. \emph{Introduction}} % Change X to a consecutive number; this 
% is for the header on each page - perhaps a shortened title

%----------------------------------------------------------------------------------------
%	SECTION 1
%----------------------------------------------------------------------------------------

\section{Motivation}

\section{Literature Review}

\subsection{Land/Ocean Warming in Global Warming}

In a transient climate the global land surface temperatures ($T_{land}$) warm 
with greater amplitude than sea surface temperatures (SSTs), leading to a 
land/sea warming contrast.  The ratio of land to sea warming tends to a value of 
around 1.5  (\citealt{Sutton2007}, \citealt{Lambert2007}, \citealt{Compo2008},
\citealt{Dommenget2009}). Previous studies have shown the land/sea warming 
contrast is not simply due to the larger heat capacity of the ocean when 
compared to land, but is a result of the dynamics of the climate system.  
\citet{Sutton2007} described an energy balance argument; assuming the anomalous 
downward surface energy flux is equal over land and ocean the land/sea warming 
contrast is caused by the difference in the partitioning of the upward energy 
flux into sensible and latent heat.  \citet*{Lambert2007} proposed that the 
stability of land/sea contrast over annual, 5 year and longer timescales is 
maintained by a land to ocean heat flux where the ability of the ocean to absorb 
the extra heat leads to a damping of $T_{land}$ variability. In this scenario 
the value of the land/sea contrast depends on the ratio of the land and sea 
climate sensitivity parameters, and can be related to the results of 
\citet{Sutton2007}.  However, as stated by \citet{Byrne2013a} the energy balance 
argument does not give a sufficient quantitative value of land warming.  
\citet{Joshi2008} proposed a conceptual model to explain how the SSTs can force 
$T_{land}$, leading to a land/sea warming contrast above unity.  There is a 
level in the atmosphere above which there is no significant land/sea contrast 
and thermal anomalies are transported efficiently around the globe. The lapse 
rate below that level is affected by temperature and moisture and different land 
and ocean lapse rates cause the land temperatures to reach an equilibrium warmer 
than the oceans.  

\citet{Dommenget2009} demonstrates the ability of oceans to cause a land/sea 
contrast on interannual and longer time scales, arguing that the asymmetric 
forcing of ocean to land is not only due to the asymmetry in area but also due 
to atmospheric water vapour feedbacks. Thus the land/sea warming contrast is a 
natural phenomena that also applies to internal interannual to decadal climate 
variability. When we think of the land/sea contrast in natural variability we 
can recognise a number of differences relative to that seen in global warming: 

Firstly, global warming is mostly a coherent warming on a global scale with a 
time evolution that is only going upwards for the relevent timescales, for 
example Compo and Sardeshmukh (2008) in figure 1b show that the pattern of 
observed surface air temperature change, calculated as the 1991--2006 average 
minus the 1961--1990 average, is largely homogeneous across the globe. In 
natural climate variability we have inhomogeneous warming and cooling patterns, 
some of them are regional others are more global, some of them have coherent 
warming and cooling (e.g. multi-pole structures) at the same time, some of them 
are closer to the land and some are over tropical warm ocean regions and others 
are over the colder extra-tropical oceans. The El Nino-Southern oscillation is 
one such mode of variability which is associated with regional warming and 
cooling \citep{Halpert1992}. When we analyse the land/sea contrast in natural 
climate variability we have to take these structures into account.

\subsection{Tropical Interannual Variability and ENSO}

When looking at the interannual variability of land and ocean the El 
Ni{\~n}o-Southern Oscillation and its teleconnections are the leading source of 
variability on a global scale. \citet*{Klein1999} discuss the concept of an 
atmospheric bridge as a method of communicating temperature anomalies from the 
equatorial Pacific to the remote tropical oceans (outside the Pacific).  
Similarly, \citet*{Chiang2002} discuss a mechanism for warming of remote 
tropical oceans during El Ni{\~n}o conditions. The tropical tropospheric 
temperature ($T_{tropos}$) increases during El Ni{\~n}o, and is largely uniform 
across the tropical strip, 20S-20N.  They attributed the amplified response over 
land to the smaller thermal inertia and reduced cooling due to evaporation.  
\citet*{Chiang2005} further found an almost instantaneous response of $T_{land}$ 
to  El Ni{\~n}o and an ocean response with a 2-3 month lag.  The ratio between 
$T_{tropos}$ and the surface warming signal was 1:1 for land but only 1:0.3 for 
oceans.  Their findings support the mechanism over oceans described by 
\citet*{Chiang2002} as holding true on the larger scale they were investigating.  
No mechanism was proposed for land warming, the higher ratio of warming was 
attributed to differing heat capacities of ocean and land. The processes of the 
atmospheric bridge responsible for the El Ni{\~n}o teleconnections are similar 
in nature to the processes of the land/sea contrast as discussed in 
\citet{Joshi2008}, suggesting that the same principles are active.

\subsubsection{The Weak Temperature Gradient Approximation}


\section{Thesis outline}

The study presented here discusses the large-scale land/sea contrast in natural 
variability, focusing on interannual timescales. We will analyse the 
characteristics of the large-scale land/sea contrast variability in observations 
and Couple General Circulation Models (CGCMs) from the CMIP5 data base. The role 
of the SST in forcing the land variability will be analysed in Atmospheric 
General Circulation Models (AGCMs) forced with observed SSTs and in a series of 
sensitivity experiments with an AGCM coupled to a slab ocean model or with fixed 
SST boundary conditions forced with different idealised SST forcings.  Our 
analysis will discuss the differences between tropical and extra-tropical 
regions.

In this article, the data and model simulations are described in the section 
2. Section 3 will discuss the evidence for the land/sea contrast larger than
unity in natural variability in observations and model simulations. This 
analysis will also explore some of the regional differences in the ocean to land 
connection. Section 4 dicsusses a series of sensitivity experiments that explore 
the role of the SST forcing, the differences between tropical and extra-tropical 
regions and that highlight the role on El Ni{\~n}o forcing.  Section 5 is an 
analysis of the mechanisms involved, illustrating how the SST forcing is 
amplified over land to result into a land/sea contrast larger than unity. In the 
final section the study will be closed with a summary and discussion.


