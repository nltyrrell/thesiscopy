\chapter{Introduction} % Main chapter title

\label{introduction} % Change X to a consecutive number; for referencing this 
% chapter elsewhere, use \ref{ChapterX}

\lhead{Chapter 1. \emph{Introduction}} % Change X to a consecutive number; this 
% is for the header on each page - perhaps a shortened title

%----------------------------------------------------------------------------------------
%	SECTION 1
%----------------------------------------------------------------------------------------

\section{Motivation}

When looking at a timeseries of annual global land surface ($T_{land}$) and sea 
surface temperatures (SST) there are two features which stand out; the two 
timeseries are largely covariant, and the land has greater variance than the 
ocean timeseries.  We define this increased variability of land surface relative 
to oceans as the interannual land/ocean thermal contrast. In a similar manner, 
for a transient climate the global $T_{land}$ warms with greater amplitude than 
the SST, leading to a global warming land/sea thermal contrast. In this study we 
will investigate the processes responsible for the interannual land/ocean 
contrast and test whether the processes that control the warming contrast are 
relevant for the positive and negative anomalies of natural variability.

\section{Land/Ocean Warming in Global Warming}\label{sec:logw}

For a global warming scenario the ratio of land to sea warming tends to a value 
of around 1.5  (\citealt{Sutton2007}, \citealt{Lambert2007}, 
\citealt{Compo2008}, \citealt{Dommenget2009}). Previous studies have shown the 
land/sea warming contrast is not simply due to the larger heat capacity of the 
ocean when compared to land, but is a result of the dynamics of the climate 
system.  \citet{Sutton2007} described an energy balance argument; assuming the 
anomalous downward surface energy flux is equal over land and ocean 
\citep{Huntingford2000} the land/sea warming contrast is caused by the 
difference in the partitioning of the upward energy flux into sensible and 
latent heat.  \citet*{Lambert2007} proposed that the stability of land/sea 
contrast over annual, 5 year and longer timescales is maintained by a land to 
ocean heat flux where the ability of the ocean to absorb the extra heat leads to 
a damping of $T_{land}$ variability. In this scenario the value of the land/sea 
contrast depends on the ratio of the land and sea climate sensitivity parameters 
and ocean heat uptake. This can be related to the results of \citet{Sutton2007};
on timescales where the land and ocean heat uptake is sufficiently small the 
land to ocean heat transport can be ignored and the value of the land/ocean 
contrast depends only on the climate sensitivity parameters, and these 
parameters are controlled by the surface energy balance.  However, as stated by 
\citet{Byrne2013a} the energy balance argument does not give a sufficient 
quantitative value of land warming, and both theories neglect the ability of 
SSTs to influence $T_{land}$.  

\subsection{The Lapse Rate Theory}
\label{intro:lapse}

\citet{Joshi2008} proposed a conceptual model to explain how the SSTs can force 
$T_{land}$ and which leads to a land/sea warming contrast above unity.  There is 
a level in the atmosphere above which there is no significant land/sea contrast 
and thermal anomalies are transported efficiently around the globe. The lapse 
rate below that level is affected by temperature and moisture. The lower 
relative humidty over land results in steeper lapse rates that cause land 
temperatures to reach an equilibrium warmer than the oceans. In addition, the 
transport of moisture over land occurs at levels much colder than the surface 
which restricts moisture supply over land, hence increased warming does not lead 
to increased evaporation, further enhancing warming. The lapse rate theory was 
extended by \citet{Byrne2013a}; where \citet{Joshi2008} assumed that the 
land/ocean contrast due to \textit{changes} in temperature disappeared at a 
certain level, \citet{Byrne2013a} make the further assumption that the absolute 
temperatures above land and ocean converge. Secondly they assume that lapse 
rates are moist adiabatic above the lifting condensation level (LCL) and dry 
adiabatic below the LCL. The land and ocean lapse rates then differ due to the 
height of the LCL, which is higher over land due to reduced moisture 
availability. The first assumption relies on the fact that the tropical 
troposphere is unable to maintain strong temperature gradients in the tropics to 
homogenise tropospheric temperatures (this is further discussed in section 
\ref{intro:wtg}), and for the second convection must be sufficient to maintain a 
moist adiabatic lapse rate, hence their theory is most appropriate in the 
tropics.

The significantly different surface properties, as well as the atmospherically 
induced balance, between land and ocean are important factors in the land/ocean 
warming contrast, but there is also an assymetry in the forcing between land and 
ocean.  \citet{Compo2008} find a strong oceanic influence to recent continental 
warming and suggest increased downward longwave radiation due to 
hydrodynamic-radiative teleconnections as a mechanism.  \citet{Dommenget2009} 
demonstrates the ability of oceans to cause a land/sea contrast on interannual 
and longer time scales.  The disproportionate forcing of land temperatures by 
the ocean is due in part to the asymmetry in area but also due to atmospheric 
water vapour feedbacks.  Modelling studies suggest that up to 86\% of historic 
anthropogenically forced change in $T_{land}$ is a response to SST changes and 
14\% is due to local forcings. \citet{Dommenget2009} suggest that on global 
scales these mechanisms are also present in the warming and cooling of 
interannual variability, as shown in detrended model runs forced only with 
historic SSTs that exhibit greater variability of global $T_{land}$ than the 
input SSTs.

Thus the land/sea warming contrast is a natural phenomena that also applies to 
internal interannual to decadal climate variability.  When we think of the 
land/sea contrast in natural variability we can recognise a number of 
differences relative to that seen in global warming: Firstly, global warming is 
mostly a coherent warming on a global scale with a time evolution that is only 
going upwards for the relevent timescales, for example \citet{Compo2008} in 
figure 1b show that the pattern of observed surface air temperature change, 
calculated as the 1991--2006 average minus the 1961--1990 average, is largely 
homogeneous across the globe. In natural climate variability we have 
inhomogeneous warming and cooling patterns, some of them are regional others are 
more global, some of them have coherent warming and cooling (e.g.  multi-pole 
structures) at the same time, some of them are closer to the land and some are 
over tropical warm ocean regions and others are over the colder extra-tropical 
oceans. The El Ni{\~n}o-Southern oscillation is one such mode of variability 
which is associated with regional warming and cooling \citep{Halpert1992}. When 
we analyse the land/sea contrast in natural climate variability we have to take 
these structures into account.

\section{Tropical Interannual Variability and ENSO}

For globally averaged $T_{land}$ and SSTs spatial scales and time scales of 
interannual variability, the El Ni{\~n}o-Southern Oscillation (ENSO) and its 
teleconnections are the leading source of variability, so it's dynamics are 
likely to play an important role in the interannual land/sea contrast.  ENSO has 
a direct effect on tropical surface and tropospheric temperatures and a 
measureable influence on global average tempereratures (REF). \citet{Klein1999} 
discuss the concept of an atmospheric bridge as a method of communicating 
temperature anomalies from the equatorial Pacific to the remote tropical oceans 
(i.e. oceans outside the Pacific).  Similarly, \citet{Chiang2002} discuss a 
mechanism for warming of remote tropical oceans during El Ni{\~n}o conditions.  
The tropical tropospheric temperature ($T_{tropos}$) increases during El 
Ni{\~n}o, and is largely uniform across the tropical strip, 20S-20N. They find 
the surface temperature responds to the tropospheric forcing with a magnitude 
relative to the mixed layer depth, and the communication of the tropospheric 
temperature is via moist convective processes. \citet{Chiang2002} attributed the 
amplified response over land to the smaller thermal inertia and reduced cooling 
due to evaporation.  \citet{Chiang2005} use model simulations of the 1997-98 El 
Ni{\~n}o to study the tropospheric temperature mechanism for tropical ENSO 
teleconnections.  They found an almost instantaneous response of $T_{land}$ to  
El Ni{\~n}o and an ocean response with a 2-3 month lag with the ratio between 
$T_{tropos}$ and the surface warming signal was 1:1 for land and 1:0.3 for 
oceans. They were interested in the uniformity of remote ocean SST response to a 
tropospheric temperature heating when the surface fluxes exhibit large 
variability, and they found latent heat flux through the boundary layer and 
clear-sky longwave radiation regulate the surface response. No specific 
mechanism was proposed for land warming, although the higher ratio of warming 
was attributed to differing heat capacities of ocean and land. In addition the 
moist static energy was found to increase more than in the boundary layer than 
the troposphere, which is not the expected result from quasi-equilibrium theory 
(e.g. \citet{Brown1997}).  Their findings support the mechanism over oceans 
described by \citet{Chiang2002} as holding true on the larger scale they were 
investigating. 

\subsubsection{The Weak Temperature Gradient approximation}
\label{intro:wtg}
The processes of the atmospheric bridge responsible for the El Ni{\~n}o 
teleconnections are similar in nature to the processes of the land/sea contrast 
as discussed in \citet{Joshi2008}, suggesting that the similar principles may be 
active. One aspect of the \citet{Joshi2008} lapse rate theory is that 
temperature anomalies are transported efficiently in the upper troposphere and 
it is well known that the tropical troposphere is unable to maintain strong 
temperature gradients.  Near the equator the small Coriolis parameter results in 
fast geostrophic adjustments by gravity waves and Kelvin waves that remove 
horizontal temperature gradients.  In the extra-tropics the quasi-geostrophic 
approximation gives us a simplified dynamical model, based on the balance of the 
pressure gradient force and rotation.  The WTG approximation is analagous and 
uses the small horizontal temperature and pressure gradients as a constraint on 
the large-scale flow and diabatic processes. The temperature equation can be 
simplified to give a balance between diabatic heating and vertical advection 
\citep{Sobel2001}. The usefulness of the WTG approximation doesn't quite match 
that of quasi-geostrophy due to the importance in the tropics of deep 
convection, which has it's own parameterization difficulties, but the WTG 
approximation allows for the parameterization of large-scale dynamics.  
\citet{Lintner2005}  investigated the applicability of the WTG approximation to 
the tropical response to ENSO with mixed results; tropospheric temperature 
perturbations were modelled realistically but there were errors in the 
precipitation response. Overall they concluded that errors in the rainfall 
response may indicate that an El Ni{\~n}o response is a poor proxy for large 
scale global warming but that the WTG approximation suitable for a low-order 
tropical response to a warming perturbation.


\section{Thesis outline}

The study presented here discusses the large-scale land/sea contrast in natural 
variability, focusing on interannual timescales. We will analyse the 
characteristics of the large-scale land/sea contrast variability in observations 
and Couple General Circulation Models (CGCMs) from the CMIP5 data base. The role 
of the SST in forcing the land variability will be analysed in Atmospheric 
General Circulation Models (AGCMs) forced with observed SSTs and in a series of 
sensitivity experiments with an AGCM coupled to a slab ocean model or with fixed 
SST boundary conditions forced with different idealised SST forcings, and our 
analysis will discuss the differences between tropical and extra-tropical 
regions. Simplified one and two dimensional models are used to further elucidate 
the tropospheric forcing of land surface and test how land surface 
characteristics affect the response. Finally the Globally Resolved Energy 
Balance model (GREB, \citet{Dommenget2011}) is used to quantify the contribution 
of different factors to regional land temperatures.

In this thesis, the data and model simulations are described in chapter 
\ref{methods}.  Chapter \ref{evidence} will discuss the evidence for the 
land/sea contrast larger than unity in natural variability in observations and 
model simulations. This analysis will also explore some of the regional 
differences in the ocean to land connection. In section \ref{mechanisms} we 
explore the mechanisms controlling the land/sea contrast with a series of 
sensitivity experiments that explore the role of the SST forcing, the 
differences between tropical and extra-tropical regions and that highlight the 
role on El Ni{\~n}o forcing and illustrate how the SST forcing is amplified over 
land to result into a land/sea contrast larger than unity. In the final chapter 
the study will be closed with a summary and discussion.
