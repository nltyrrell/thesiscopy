\chapter{Conclusion} 

\label{conclusion} 

\lhead{Chapter 5. \emph{Conclusions}} 

%----------------------------------------------------------------------------------------
%	Summary and Discussion
%----------------------------------------------------------------------------------------
\section{Summary and discussion}

The aim of this study was to analyse the large-scale land/sea warming and 
cooling contrasts in natural variability in observations and model simulations.  
Comparing the statistics between observations, coupled climate model simulations 
and idealised atmosphere-only SST forced simulations, we found some consistent 
characteristics of the land/sea contrast, estimated the role of the SST in 
forcing the land and described the main tropical forcing and amplification 
mechanism for the tropical SST to influence $T_{land}$.

The observations, CGCM simulations from the CMIP5 models and AMIP-type forced 
AGCM experiments show a quite consistent picture for the tropical and global 
land/sea interaction. $R_{L/S}$ is larger than unity on both a tropical and a 
global scale. The global $R_{L/S}$ tends to be larger than any zonal band, 
suggesting that the land/sea warming and cooling contrast in natural variability 
is stronger on the larger-scale. However, substantial regional differences exist 
in this. In particular, in the extra-tropical regions the $R_{L/S}$ tends to be 
smaller or insignificant. We also find some disagreement in the Northern 
hemisphere extra-tropics with the observations showing a significant land/sea 
correlation that doesn't exist in the CGCM simulation.  However, it is unclear 
from the analysis whether this points towards a model problem or an 
observational data problem.


An important part of this study was determining causality in the land/sea 
relationship. This was investigated with AMIP runs and sensitivity experiments.  
Forcing an AGCM model with observed SSTs results in a realistic land/sea 
contrast in the tropics, while in the Northern Hemisphere extra-tropics the 
value differed from observations but was still similar to coupled models. This 
can indicate that: either the observed covariance between land and ocean is not 
SST forced and comes from internal atmospheric variability or a land to ocean 
feedback exists, which clearly will be missing from AMIP runs. The atmosphere in 
the extra-tropics is known to generate most of the SST variability in the 
extra-tropics (e.g. \citealt{Hasselmann1976}, \citealt{Barsugli1998}, 
\citealt{D.2002}) with only a weak feedback to the atmospheric variability 
(\citealt{Barsugli1998}).  These results suggest that AMIP type simulations will 
not cause much low-frequency atmospheric variance in the extra-tropics forced 
from extra-tropical SST.  However \citet{Folland2005} demonstrate that an SST 
forced model is capable of simulating large-scale land surface air temperature 
variance.

These uncertainties in the extra-tropical regions of SST forced runs shouldn't 
be present in coupled Coupled models, and assuming the observed strong $R_{L/S}$ 
in the Northern Hemisphere extra-tropics is real, the lack of a strong $R_{L/S}$ 
in the CMIP CGCM simulations either suggests that the correct atmosphere-ocean 
interaction is missing or indicates that the CGCM simulations do not produce the 
right kind of natural SST variability.  The latter may indeed be a problem, as 
it has been shown that the simulated modes of SST variability in the 
extra-tropical oceans in the CMIP5 CGCM simulations are indeed quite different 
from the observed \citep{Wang2014}. In this context it may be useful in 
continuing studies to conduct AMIP simulations with the SST variability from the 
different CMIP5 simulations, which may help in understanding the differences 
towards the observed.

An interesting aspect of the tropical connection to $T_{land}$ is the relatively 
small correlation with the NINO3 SST index and the role of the remote tropical 
oceans in the response of $T_{land}$. The slow response of the Indian and 
Atlantic tropical basins to the Pacific ocean forcing leads to the delay of the
$T_{land}$ response to the NINO3 SST index by several months (\citealt{Lau1996}, 
\citealt{Chiang2005}, \citealt{Su2005}).  In addition to the delay, the combined 
Pacific/remote ocean forcing further amplifies the $T_{land}$ response.  With 
the help of the idealised ENSO-like experiments we confirmed that the delayed 
land response is due to the slowly responding remote tropical oceans and this 
leads to increased variability of $T_{land}$.  The process of how $T_{land}$ is 
being forced by ENSO can be outlined as follows: the NINO3 SST anomalies in the 
tropical Pacific are transported via the troposphere and land responds without 
delay, the remote tropical oceans respond on a timescale of 4--6 months, and 
tropical land also responds quickly to this delayed forcing which leads to a 
peak in the land's response to ENSO at a delay of 3 months.

The large sensitivity (amplification) of $T_{land}$ to tropical ocean 
temperature anomalies is due to the enhanced upper level atmospheric warming 
that goes along with tropical SST variability. The latent heat released by moist 
convection leads to upper level temperature variations that are larger in 
amplitude than the source SST anomalies. The amplified positive and negative 
anomalies are transported to land, leading to an increase in temperature 
variability over land compared to oceans. This mechanism is essentially the same 
as that proposed for explaining the equilibrium global warming land/sea warming 
contrast (e.g.  \citet{Joshi2008}, \citet{Dommenget2009} or \citet{Byrne2013}).  
The link via the upper level amplification by moist convection suggests that the 
climate will be more sensitive to SST variability in warm ocean regions that 
allow for increases in deep convections. The processes we explained don't extend 
to the extra-tropics due to the lack of strong large-scale moist convection, and 
as such we don't fully explain extra-tropical values of the land/sea contrast.  
However the Northern Hemispheric correlation values seen in observations, and 
the non-linear model response of the extra-tropical continents to tropical and 
extra-tropical ocean forcings indicate that the land/sea connection outside of 
the tropics is more subtle but still important.


%----------------------------------------------------------------------------------------
%	Conclusions
%----------------------------------------------------------------------------------------

\section{Conclusions}


