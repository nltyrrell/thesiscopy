\chapter{Sensitivity Experiments} 

\label{experiments} 

\lhead{Chapter 5. \emph{Experiments}} 

In the previous section we illustrated that the SST variability is forcing an 
amplified response in the $T_{land}$ variability. It was also shown that the 
link to tropical SSTs was much stronger than the link to extra-tropical SSTs.  
This result suggests that the atmosphere and land are more sensitive to tropical 
SST variability, but it may also illustrate that tropical SST variability is 
stronger than extra-tropical or may have patterns of variability that affect the 
land more strongly than those from the extra-tropics. In the first set of 
sensitivity experiments we explore the differences between tropical and 
extra-tropical SST forcing and in the second set of sensitivity experiments we 
take a closer look at ENSO SST variability, which is the main driver of global 
SST variability.

%----------------------------------------------------------------------------------------
%	ACCESS
%----------------------------------------------------------------------------------------

\section{ACCESS SST perturbation experiments}

In order to address the sensitivity of the atmosphere and land to identical SST 
anomalies from the tropical or extra-tropical regions we conduct a series of 
idealised sensitivity experiments, with homogeneous increases in the SST by +1K. 
These experiments are similar to some of the classical SST response experiments 
done in previous studies in the context of global warming or climate sensitivity 
(\citealt{Cess1990}, \citealt{Dommenget2009}, \citealt{Compo2008}).

%-----------------------------------
%	SUBSECTION 1
%-----------------------------------
\subsection{AMIP}

%-----------------------------------
%	SUBSECTION 2
%-----------------------------------
\subsection{+1K experiments}

Figure \ref{fig:1K} shows the surface temperature response (control removed) 
from the +1K experiments; where +1K was added to the oceans in the tropics, 
extra-tropics or globally. In response to a tropical SST perturbation there is a 
large tropical response, greatest over equatorial South America and Africa, 
India and the maritime continent (Fig.\ref{fig:1K} a). The tropical +1K ocean 
perturbation leads to $T_{land}>+1$K in most tropical areas. Thus the SST 
forcing is amplified. The extra-tropical land the response to the tropical 
forcing is not significant everywhere, but some regions also show an amplified 
response to the tropical SST forcing (e.g. central Asia and parts of Europe and 
North America).  An extra-tropical $T_{land}$ response to tropical SST is seen 
for seasonal averages in the winter months of each hemisphere, the Northen 
Hemsiphere winter response is shown in Figure \ref{fig:1K}e.

When looking at the annual mean response of $T_{land}$ to extra-tropical SST 
perturbations there is little significant response, however for seasonal 
averages both hemispheres show a significant response in their respective winter 
months, shown for the Northern Hemisphere in Figure \ref{fig:1K} e-g. The 
response of $T_{land}$ is also amplified in some regions relative to the initial 
perturbation.  However, the extra-tropical forcing again leads to a weaker land 
response than the tropical forcing, as was also found in the AMIP simulations.  
Also similar to the AMIP simulations we again find that the  global SST forcing 
has a bigger impact than the tropical only forcing for the annual mean. In 
addition the response of the global SST forcing is greater than the 
superposition of the tropical and extra-tropical forcing (comparing 
Fig.\ref{fig:1K} c and d).  This again indirectly suggests that the 
extra-tropical SST forcing does lead to a significant land response.

%-----------------------------------
%	SUBSECTION 3
%-----------------------------------
\subsection{Half-AMIP}

%-----------------------------------
%	SUBSECTION 4
%-----------------------------------
\subsection{Pacemaker Experiments}


On inter-annual timescales ENSO is the most significant global climate driver. 
It is therefore remarkable that in the analysis of the CMIP5 model simulations 
the NINO3 region did not show up with a high correlation to global $T_{land}$ 
(see Fig.\ref{fig:cmip_cormap}). The ENSO region in the tropical Pacific has a 
lower correlation with $T_{land}$ than adjacent regions and the other ocean 
basins. Using the combined monthly mean CMIP5 surface temperature anomalies, 
Fig.\ref{fig:xcor} e) shows the lagged correlations between NINO3 SST and global 
$T_{land}$. The NINO3 region is seen to lead global land by 4 months.  Typically 
land has a fast response time to forcings, which would not result in a 
4 month delay, so this result suggests that the full land response is not 
directly forced by the NINO3 SST but is most likely caused by something else. 
This other forcing may be delayed to the ENSO variability by about 4 months. 
Since we have seen in Fig.\ref{fig:cmip_cormap} that global $T_{land}$ is highly 
correlated to other tropical ocean SST, it seems likely that $T_{land}$ is 
linked to the slower ocean response in the remote tropical oceans and not 
directly to the NINO3 region.

To address this question we conducted a series of idealised ENSO-response 
experiments. In the first experiment we prescribe an oscillating ENSO pattern (a 
regression between NINO3 and SSTs shown in Fig.\ref{fig:regpat}) in the tropical 
Pacific and fixed SST climatologies elsewhere.  The oscillation period of the 
ENSO signal is 4 years, peaking in January. In the second experiment we allow 
SST variability outside the tropical Pacific simulated by a simple slab ocean 
model.  Thus, in the second experiment the global ocean SSTs can respond to the 
oscillating ENSO pattern forcing.

Figure \ref{fig:xcor} (i-l) shows cross-correlations from the ENSO-FIXSST and 
ENSO-Slab forcing experiments. In i) and j) we see that for the fixed SST 
experiment the global and tropical land responds to the ENSO-like forcing (red 
line), and does so without the delay seen in Figure \ref{fig:xcor} e). When a 
slab ocean is introduced the land responds with a realistic delay of around 4 
months.  The peak slab ocean response is at 6 months, implying that the land is 
responding immediately to the initial Pacific ocean forcing and then 
subsequently to the delayed slab ocean response. The delayed land response is 
also associated with a higher correlation to the NINO3 region. Comparing the 
global and tropical averages, the main difference is the magnitude of the peak 
correlation, but in the tropics the slab ocean also results in the peak land 
correlation being higher than the peak ocean correlation. So the delayed 
response of the remote tropical oceans to a Pacific ocean forcing explains both 
the delayed land response and some part of the amplification of the oceanic 
temperature signal over land. In the extra-tropics there is only a very weak 
influence of the ENSO forcing on land temperatures in the sensitivity 
experiments (Fig.  \ref{fig:xcor} g, h), and the tropical Pacific has little 
influence on the slab ocean in the extra-tropics. The observations and CMIP5 
models also don't show a significant relationship between the extra-tropics and 
NINO3.

%----------------------------------------------------------------------------------------
%	RCM - Single Column
%----------------------------------------------------------------------------------------

\section{Radiative Convective Model - Single Column Configuration}

\subsection{Experimental Design}
\paragraph{explain setup}
\begin{itemize}
	\item WTG assumption
	\item Generation of atmospheric profiles
	\item Parameter values;
		\begin{itemize}
			\item Land surface - Evaporative fraction
			\item WTG - pressure level of fixed temp profile
			\item Interactive clouds
		\end{itemize}
\end{itemize}
\subsection{Results}
\paragraph{Values of l/s contrast for varying land/tropos parameters}
\paragraph{Atmospheric profiles}
\paragraph{Variability and instability - Role of convection}
\subsection{Conclusions}

%----------------------------------------------------------------------------------------
%	RCM - Two Column
%----------------------------------------------------------------------------------------

\section{Radiative Convective Model - Two-Dimensional Configuration}

\subsection{Experimental Design}

\paragraph{explain setup}
\begin{itemize}
	\item Land/Ocean column configurations
	\item Parameter values;
		\begin{itemize}
			\item Land surface - Evaporative fraction
			\item Interactive clouds
		\end{itemize}
\end{itemize}


\subsection{Results}

\paragraph{Values of l/s contrast for varying land parameters}
\paragraph{Atmospheric profiles in temp, RH, etc}
\paragraph{Variability and instability - Role of convection and clouds}
\paragraph{Circulation response}


\subsection{Comparison of RCM and GCM results}

\paragraph{Generation of suitable l/s contrast at varying latitudes/evap fraction}
\paragraph{Mapping results}
\paragraph{Differences/Similarities between GCM/RCM}


\subsection{Conclusions}

%----------------------------------------------------------------------------------------
%	GREB 
%----------------------------------------------------------------------------------------

\section{GREB Model}

\subsection{Deconstructing GCM response}
\paragraph{motivation}
\paragraph{Discuss variables used/available}
\paragraph{Response of pacemaker exp.}
\paragraph{Response of amip exp.}

%-----------------------------------
%	Pacemaker
%-----------------------------------

\subsection{Results - Pacemaker}
\paragraph{u, v winds}
\paragraph{Cloud cover}
\paragraph{Soil Moisture}
\paragraph{SST}
\paragraph{Combined response}


%-----------------------------------
%	SUBSECTION 2
%-----------------------------------
\subsection{AMIP Experiments}

\paragraph{u, v winds}
\paragraph{Cloud cover}
\paragraph{Soil Moisture}
\paragraph{SST}
\paragraph{Combined response}