\chapter{Data and Methods} % Main chapter title

\label{methods} % Change X to a consecutive number; for referencing this chapter 
% elsewhere, use \ref{ChapterX}

\lhead{Chapter 2. \emph{Data and Methods}} % Change X to a consecutive number; 
% this is for the header on each page - perhaps a shortened title

This chapter will discuss the observational and model data, the models and the 
experiments conducted with them, and some of the statistical methods applied to 
the data.

%----------------------------------------------------------------------------------------
%	DATA
%----------------------------------------------------------------------------------------

\section{Data}

This section will outline the data used for this thesis, including processing 
performed, a discussion of uncertainties for each data set and any inherent 
limitations.
%-----------------------------------
%	OBSERVATIONS
%-----------------------------------

\subsection{Observational Data}

\paragraph{Hadley Centre Data - Crutemp, Hadsst, Hadisst}
FROM PAPER: The observational surface temperature datasets used were the 
Climatic Research Unit Temperature data set, version 4 (CRUTEM4) 
\citep{Brohan2006} for $T_{land}$ and the Hadley Centre SST data set, version 2 
(HadSST2) \citep{Rayner2006}for the SST, $T_{ocean}$.  Temperature data previous 
to 1950 was excluded in the analysis of the land/sea interactions as the smaller 
data coverage area can cause errors in the statistical comparison of the two 
datasets \citep{Dommenget2009}.

\begin{itemize}
	\item Three data products were used from the Hadley centre for surface 
		temperatures;
	\item Hadisst, hadsst2, Crutemp (need version numbers here)
	\item Hadsst2
	\begin{itemize}
		\item Global gridded SST dataset based on observations
		\item Data coverage decreases with time.
		\item Prior to 1950 data coverage is too sparse to allow strong 
			statistical conclusions \citep{Dommenget2009} when compared with 
			Crutemp.
		\item This dataset, along with Crutemp, was used for observations as it 
			was considered the truest representation, as there is no 
			interpolation or processing done on the data; other than that to 
			correct inconsistencies.
		\item Data coverage:
	\end{itemize}
	\item Crutemp
	\begin{itemize}
		\item Global gridded land temperature dataset based on observations
		\item Data coverage: Large areas without data in the tropics, particularly Africa and South America.
		\item GCM experiments show that these areas tend to be important in 
	terms of the land response to tropical SST perturbations. The effect of this 
	missing data is investigated further in Chapter 5.
	\end{itemize}
	\item Hadisst
	\begin{itemize}
		\item Global gridded SST dataset based on observations, and interpolated in regions without data.
		\item Data coverage decreases with time.
		\item The data prior to 1950 is still used for AMIP runs because even 
			though it is unsuitable for comparing the land and ocean timeseries	
			it is suitable as a representation of realistic SST variability. For 
			the AMIP runs we want to see how the land surface responds to SST 
			variability and the interpolated SST data is suitable for this.
	\end{itemize}
\end{itemize}

\paragraph{Microwave Sounding Unit}
\begin{itemize}
	\item Description, references
	\item Temporal, areal coverage
	\item Known issues.
	\item Reason for use, relationship to l/o contrast
\end{itemize}

%-----------------------------------
%	NCEP
%-----------------------------------

\subsection{Reanalysis Data}

\paragraph{NCEP Data}
\begin{itemize}
	\item Forecast model that assimilates past observational data.
	\item Length is limited to satellite era, post 1979.
	\item Used in initial investigations as alternative to Hadley centre 
		observations.
\end{itemize}

%-----------------------------------
%	CMIP
%-----------------------------------

\subsection{Coupled Model Data}

\paragraph{CMIP5}
For the analysis of CGCMs we used all available pre-industrial control runs from 
the CMIP5 datasets \citep{Taylor2012}, see Table \ref{tab:cmip5}. 100 years from 
each of the 35 models was used, and anomalies were defined seperately for each 
model relative to its climatology. The multi-model mean of the CMIP5 data was 
calculated using the anomalous (relative to each model’s climatology) timeseries 
from each model, these were combined end-to-end to generate a 3500 year 
timeseries. 



%----------------------------------------------------------------------------------------
%	MODEL EXPERIMENTS
%----------------------------------------------------------------------------------------

\section{Model Experiments}

This section will describe the models used, and the experiments performed with 
each model.

%-----------------------------------
%	SUBSECTION ACCESS
%-----------------------------------

\subsection{ACCESS Model}

\paragraph{Description of ACCESS}
The sensitivity experiments were performed with the UK Meteorological Office 
Unified Model AGCM with HadGEM2 atmospheric physics (\citet{Davies2005}; 
\citet{Martin2010}; \citet{Bellouin2011}) at an atmospheric resolution of N48 
($3.75^{\circ} \times 
2.5^{\circ}$). This was forced with prescribed SSTs or a slab ocean. The slab
ocean assumes a constant mixed layer depth of 50 metres and is forced by flux 
correction terms to have on average the HadISST 1950--2010 SST climatology 
\citep{Wang2014}. A constant mixed layer was chosen in part for simplicity as 
this model setup is intended to be of intermediate complexity, between a simple 
prescribed SST run and a fully coupled dynamic ocean. We use a slab ocean as 
simple way of modelling SST variability, and while we would expect that the 
response time of the remote oceans would scale with mixed layer depth 
(\citealt{Su2005a}, \citealt{Lintner2007}), the constant 
50 metre assumption gives a first order but realistic response.

\paragraph{AMIP Run}
Three primary types of experiments were conducted: AMIP-type; sensitivity to 
mean SST increases; and El Ni{\~n}o pattern forcing experiments, see Table 
\ref{tab:senseexp}. The latter being similar to a `pacemaker' experiment, as 
described by ~\citealt{Alexander1992}, ~\citealt{Alexander1992a}, 
~\citealt{Lau2000} and ~\citealt{Lu2011}. The AMIP type runs used HadISST from 
1870 to 2010. While it was determined that HadSST2 data was only suitable
from 1950 for calculating statistics relating to the observational value of the 
land/sea contrast, the HadISST data was used from 1870. The reason is that the 
HadISST SSTs were primarily used as a representation of realistic interannual 
variability with which to force the atmospheric model.  In the AMIP runs the SST 
and $T_{land}$ are consistent because $T_{land}$ responds to the prescribed SST.  
Whether or not these were the true SST values in the real world is not that 
important in the context of this study.  We use the early SST values from 
HadISST to generate realistic SST variability; thus we use the earlier period to 
get more statistics. In the statistical analysis of observed SST vs. land the 
errors in the observed SST do matter as they no longer co-vary with the land, so 
we cannot use the early SST and Tland values. 

\paragraph{Half-AMIP/Half-1K experiments}
Simulations forced with idealised SST patterns used a 12 month climatology of 
the HadISST data from 1950-2010 as the reference control climate. The division 
between tropics and extra-tropics for these experiments was chosen to be 
$28^{\circ}$N/S, with the tropical forcing applied to the oceans in the zonal 
band bordered by $28^{\circ}$N/S, and the extra-tropical forcing applied from 
$28^{\circ}$N/S to the poles. For the model resolution used this most closely 
divides the oceans in half by area, with slightly more area in the 
extra-tropics.

\paragraph{Pacemaker experiment}
For the El Nino pattern forcing experiments a canonical El Nino pattern was 
generated using HadISST monthly mean data from a linear regression between NINO3 
and SSTs, as shown in Figure 1. This pattern was imposed in the tropical Pacific 
between $30^{\circ}$N/S and $155^{\circ}$E to the eastern boundary of the 
Pacific, on to monthly climatological SSTs (i.e. a repeating 12 month seasonal 
cycle).  The values of the anomaly was based on the regression values, with a 
maximum temperature anomaly of 1.41K.  The pattern was oscillated with a period 
of 4 years, peaking in January.  Outside of the tropical Pacific there were two 
scenarios; fixed SSTs using the HadISST 1950-2010 climatology, and the slab 
ocean. 


%-----------------------------------
%	SUBSECTION RCM
%-----------------------------------

\subsection{Radiative Convection Model}

\paragraph{Model Description}
\begin{itemize}
	\item MIT model, references
\end{itemize}

\paragraph{Model Setups}
\begin{itemize}
	\item Geometry, 1D, 2D
	\item Forcing options; SST, CO2, WTG
	\item Parameter options; land surface, model levels, interactive clouds
\end{itemize}

\paragraph{Experiments}
\begin{itemize}
	\item 1D, WTG setup
	\item 2D, SST forced
	\item 2D, CO2 forced
	\item Parameter experiments
\begin{itemize}
	\item Surface properties
	\item Interactive Clouds
\end{itemize}
\end{itemize}

%-----------------------------------
%	SUBSECTION GREB
%-----------------------------------

\subsection{Globally Resolved Energy Balance Model}

\paragraph{Model Description}
\begin{itemize}
	\item GREB model, references
\end{itemize}


\paragraph{Experiments}
\begin{itemize}
	\item AMIP Forcings
	\item Pacemaker forcings
\end{itemize}

%----------------------------------------------------------------------------------------
%	SECTION STATS
%----------------------------------------------------------------------------------------

\section{Statistical methods}

This section will describe the statistical methods applied to the data


\subsection{Regression of land/ocean}


All further analysis is based on annual mean anomalies with one exception in 
Section 4.2, which is based on monthly mean anomalies for monthly mean lag-lead
correlations. The land/sea contrast, $R_{L/S}$, is defined by the following 
regression model;

\begin{equation}
T_{land} = R_{L/S} \cdot T_{ocean}
\end{equation}
where
\begin{equation}
	R_{L/S} = \rho_{land,ocean}\cdot \frac{\sigma_{land}}{\sigma_{ocean}}
\end{equation}

With $T_{land}$ and $T_{ocean}$ as the annual mean surface temperature anomalies 
of areal-averaged land and ice-free oceans (global or zonal average), 
respectively, $\rho_{land,ocean}$ is the correlation coefficient between 
$T_{land}$ and $T_{ocean}$, and $\sigma_{land}, \sigma_{ocean}$ are the standard 
deviations of $T_{land}, T_{ocean}$. While regression coefficients are often 
used to measure co-variability, $R_{L/S}$ can also be thought of as an 
amplification factor; how much is the temperature of land amplified compared to 
the SST. A simple ratio, as for the land/sea contrast in global warming, is 
unsuitable as we are dealing with anomalies varying around zero. As a simple 
example, if the anomalous SSTs are +1K the first year and -1K the second, and 
the land temperature is +1.5K then -1.5K respectively, we have a value of 
$R_{L/S} = 1.5$. In this way we consider the regression coefficient an 
appropriate measure of the land/sea temperature contrast in interannual 
variability.
