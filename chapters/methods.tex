\chapter{Data and Methods} % Main chapter title

\label{methods} % Change X to a consecutive number; for referencing this chapter 
% elsewhere, use \ref{ChapterX}

\lhead{Chapter 2. \emph{Data and Methods}} % Change X to a consecutive number; 
% this is for the header on each page - perhaps a shortened title

In this chapter we discuss the observational and model data, the models and the 
experiments conducted with them, and some of the statistical methods applied to 
the data.

%----------------------------------------------------------------------------------------
%	DATA
%----------------------------------------------------------------------------------------

\section{Data}

This section will outline the data used for this project, including processing 
performed, a discussion of uncertainties for each data set and any inherent 
limitations.

\subsection{Observational Data}

Three data products from the UK Met Office Hadley centre and the University of 
East Anglia Climatic Research Unit were used as surface temperatures 
observations; Climatic Research Unit land Temperature data set, version 
4 (CRUTEM4) \citep{Brohan2006}, the Hadley Centre SST data set,
version 2 (HadSST2) \citep{Rayner2006} and the Hadley Centre global sea ice and 
and SST, version 1 (HadISST) (ref).

HadSST2 is global gridded SST dataset based on observations, with the data 
coverage decreasing with time. Prior to 1950 data coverage is too sparse to 
allow strong statistical conclusions \citep{Dommenget2009}, and measuring the 
land/sea interactions by comparison with CRUTEM4 can lead to errors. This 
dataset, along with CRUTEM4, was used for observations as it was considered the 
truest representation, as there is no interpolation or processing done on the 
data; other than that to correct inconsistencies. (Data coverage discussion?)

CRUTEM4 is a global gridded land temperature dataset based on observations.  
Regarding the data coverage; there are large areas without data in the tropics, 
particularly in tropical Africa and Amazonian South America. Global climate 
model experiments show that these areas may be important in terms of the land 
response to tropical SST perturbations, as such the effect of this missing data 
is investigated further in Chapter \ref{evidence} by masking the model results 
to simulate the data coverage of the observations and testing the robustness of 
the statistics used to measure the land/sea contrast. It is found the data 
coverage issues may effect the magnitude of some results but not the 
conclusions.

HadISST1 is a global gridded SST dataset based on observations, and interpolated 
in regions without data. It uses the same sources as HadSST2 but as missing data 
points are interpolated there is no decreases in coverage with time. The 
increased amount of interpolation means there is no increase in information so 
the dataset is no more useful than HadSST2 when comparing to land temperatures.  
Instead, the HadISST1 data is used as the input for model runs. The data prior 
to 1950 is still used for AMIP-type experiments (see \ref{methods:amip}) because 
even though it is unsuitable for comparing the land and ocean timeseries	it 
is suitable as a representation of realistic SST variability. For the AMIP runs 
we want to see how the land surface responds to SST variability and the 
interpolated SST data is suitable for this.

All observational data was linearly detrended, as linear trends strongly 
influence correlations and would lead to spurious results. However variability 
on decadal timescales was not removed. The use of highpass filters on the data, 
which could have the effect of removing decadal length oscillations, lead to 
unrealistic results, and it was decided that simple linear detrend would be 
sufficient to remove the long-term variability most affecting any correlations 
and regressions with the least amount of processing.

\paragraph{Microwave Sounding Unit}
\begin{itemize}
	\item Description, references
	\item Temporal, areal coverage
	\item Known issues.
	\item Reason for use, relationship to l/o contrast
\end{itemize}


\subsection{Reanalysis Data}

The NCEP-Reanalysis dataset uses a forecast model (ref, details) that 
assimilates past observational data as boundary conditions. The result is a high 
spatial and temporal resolution gridded dataset of surface and tropospheric 
variables, as close to observations as possible.  Only satellite data is used 
which mean the length is limited to satellite era, post 1979. The NCEP data was 
Used in initial investigations as alternative to Hadley Centre observations.

\subsection{Coupled Model Data}

For the analysis of CGCMs we used all available pre-industrial control runs from 
the CMIP5 datasets \citep{Taylor2012}, see Table \ref{tab:cmip5} for the full 
list. The pre-industrial control runs, which lack any anthropogenic or natural 
radiative forcings (i.e. no changes in $CO_2$, volcanoes, etc)  were chosen in 
order to study the unforced internal variability. 100 years from each of the 35 
models was used, and anomalies were defined seperately for each model relative 
to its climatology. The multi-model mean of the CMIP5 data was calculated using 
the anomalous (relative to each model’s climatology) timeseries from each model, 
these were combined end-to-end to generate a 3500 year timeseries. 

\begin{center}
	\begin{table}[h]
	\caption{CMIP5 models used in this study. 100 years of the piControl run was 
	used from each model.}
		\label{tab:cmip5}
		\tiny
	\begin{tabular}{ l  l  l  l l}
		Originating Group(s) & Country & Model \\ \hline
		CSIRO and BOM & Australia & ACCESS1.0 \\
Beijing Climate Center, China Meteorological Administration & China & BCC-CSM1.1 
		 \\
Beijing Climate Center, China Meteorological Administration & China & 
		BCC-CSM1.1-m \\
		GCESS, Beijing National University & China & BNU-ESM \\
National Center for Atmospheric Research & USA & CCSM4 \\
		National Center for Atmospheric Research  & USA & CESM1-BGC  \\
		National Center for Atmospheric Research  & USA & CESM1-CAM5  \\
		National Center for Atmospheric Research  & USA & CESM1-FASTCHEM \\
		National Center for Atmospheric Research  & USA & CESM1-WACCM  \\
		Centro Euro-Mediterraneo per i Cambiamenti & Italy & CMCC-CM \\
		Centro Euro-Mediterraneo per i Cambiamenti & Italy & CMCC-CMS \\
		CSIRO and QCCCE & Australia & CSIRO-Mk3-6-0 \\
Meteo-France/Centre National de Recherches Meteorologiques & France & CNRM-CM5 
		\\
Canadian Centre for Climate Modelling and Analysis & Canada & CanESM2 \\
		Institute of Atmospheric Physics and Chinese Academy of Sciences & China & 
		FGOALS-g2 \\
		Institute of Atmospheric Physics and Chinese Academy of Sciences & China & 
		FGOALS-s2 \\
		The First Institution of Oceanography & China & FIO-ESM \\

Geophysical Fluid Dynamics Laboratory & USA & GFDL-CM3 \\
Geophysical Fluid Dynamics Laboratory & USA & GFDL-ESM2G \\
Geophysical Fluid Dynamics Laboratory & USA & GFDL-ESM2M \\
NASA / Goddard Institute for Space Studies & USA & GISS-E2-H \\
NASA / Goddard Institute for Space Studies & USA & GISS-E2-R \\
Hadley Centre for Climate Prediction and Research/Met Office & UK & HadCM3 \\
Hadley Centre for Climate Prediction and Research/Met Office & UK & HadGEM2-CC 
 \\
Hadley Centre for Climate Prediction and Research/Met Office & UK & HadGEM2-ES 
 \\
Institute for Numerical Mathematics & Russia & INM-CM4 \\
Institut Pierre Simon Laplace & France & IPSL-CM5A-LR \\
Institut Pierre Simon Laplace & France & IPSL-CM5A-MR \\
Institut Pierre Simon Laplace & France & IPSL-CM5B-LR \\
		Atmosphere and Ocean Research Institute (AORI),    & Japan & MIROC5 \\
		\	National Institute for Environmental Studies (NIES) and  &&& \\
		\	Japan Agency for Marine-Earth Science and Technology (JAMSTEC) && & \\
AORI, NIES and JAMSTEC		& Japan & MIROC-ESM \\
Max Planck Institute for Meteorology & Germany & MPI-ESM-LR \\
Max Planck Institute for Meteorology & Germany & MPI-ESM-P \\
Max Planck Institute for Meteorology & Germany & MPI-ESM-MR \\
Meteorological Research Institute & Japan & MRI-CGCM3 \\
Norwegian Climate Centre & Norway & NorESM1-M \\
Norwegian Climate Centre & Norway & NorESM1-ME \\
	\end{tabular}
	\end{table}
\end{center}


%----------------------------------------------------------------------------------------
%	MODEL EXPERIMENTS
%----------------------------------------------------------------------------------------

\section{Model Experiments}

This section will describe the models that were used, and the core experiments 
performed with each model.


\subsection{ACCESS Model}

\subsubsection{Description of ACCESS}
The sensitivity experiments were performed with the UK Meteorological Office 
Unified Model AGCM with HadGEM2 atmospheric physics (\citet{Davies2005}; 
\citet{Martin2010}; \citet{Bellouin2011}) at an atmospheric resolution of N48 
($3.75^{\circ} \times 2.5^{\circ}$). The model was either forced with prescribed 
SSTs or contained a slab ocean. 

Three primary types of experiments were conducted: AMIP-type; sensitivity to 
mean SST increases or decreases; and El Ni{\~n}o pattern forcing experiments, 
see Table \ref{tab:senseexp} for full details. The latter El Ni{\~n}o forcing 
experiment is similar to a `pacemaker' experiment, as described by 
~\citealt{Alexander1992}, ~\citealt{Alexander1992a}, ~\citealt{Lau2000} and 
~\citealt{Lu2011}. 

\begin{center}
	\begin{table}[h]
		\caption{Idealised model simulations discussed in this study. Atmospheric 
		component was HadGEM2 at N48 resolution.}
		\label{tab:senseexp}
		\scriptsize
	\begin{tabular}{ l  l  l  l l}
		\textit{Name}		&  \textit{Ocean} & \textit{Time} &\textit{ Notes }\\ \hline
	AMIP-global &    HadISST & 1870-2012 &  \\
	AMIP-tropics	   & Tropics: HadISST & 1870-2012 & Climatological SSTs with \\
		&Extra-tropics: FIXSST&& anomalies applied in tropics \\
	AMIP-extra-tropics     & Extra-tropics: HadISST & 1870-2012 & 
		Climatological SSTs with \\
		& Tropics: FIXSST && anomalies applied in extra-tropics\\
		FIXSST    & Climatology & 100 years & Climatological SSTs based  \\
								  &&& on HadISST 1950-2013 \\
		+1K Global    & FIXSST,  & 100 years & Climatology with \\
												 & +1K && +1K added to global oceans \\
		+1K Tropics     & FIXSST, & 100 years & Climatology with \\
												 & +1K in Tropics && +1K added to 
		tropical oceans \\
		+1K Extra-tropics	   & FIXSST & 100 years &Climatology with \\
											& +1K in Extra-tropics && +1K added to 
		extra-tropical oceans \\
	Slab 	   & 50m mixed layer ocean & 100 years & \\
	ENSO-FIXSST    &  FIXSST, & 100 years & Climatology with \\
											& El Ni{\~n}o pattern && oscillating 
		pattern in tropical Pacific\\
	ENSO-slab    & Slab,  & 100 years & 50m mixed layer ocean with \\
											& El Ni{\~n}o pattern && oscillating 
		pattern in tropical Pacific\\
	\hline
	\end{tabular}
	\end{table}
\end{center}


\subsubsection{AMIP-Type experiments}\label{methods:amip}
AMIP (Atmosphere-only Model Inter-comparison Project, (ref)), much like CMIP, is 
a project to compare results from many modelling centres, except instead of 
coupled ocean-atmosphere models the models are forced with observational SSTs.  
Therefore an we consider an AMIP-type experiment one where a model is forced 
with observational, or similar to observational SSTs. There are a few benefits 
to this setup; it allows us to test the effect of a range of SST anomalies on 
continental temperatures, it is computationally much cheaper than a coupled run 
and doesn't require long spin up times, and the contained boundary conditions 
stabilise the model and prevent drift. The key limitation of AMIP runs is the 
loss of atmosphere to ocean feedbacks which are potentially important for the 
eveolution of both systems. This downside is partially addressed with the use of 
a mixed-layer slab ocean, discussed in section \ref{methods:pacemaker}.

The AMIP type runs used HadISST from 1870 to 2010. While it was determined that 
HadSST2 data was only suitable from 1950 for calculating statistics relating to 
the observational value of the land/sea contrast, the HadISST data was used from 
1870. The reason is that the HadISST SSTs were primarily used as a 
representation of realistic interannual variability with which to force the 
atmospheric model.  In the AMIP runs the SST and $T_{land}$ are consistent 
because $T_{land}$ responds to the prescribed SST.  Whether or not these were 
the true SST values in the real world is not important in the context of this 
study.  We use the early SST values from HadISST to generate realistic SST 
variability; thus we can use the earlier period to get more statistics. In the 
statistical analysis of observed SST verse land the errors in the observed SST 
do matter as they no longer co-vary with the land, so we cannot use the early 
SST and $T_{land}$ values.

\subsubsection{Half-AMIP/Half-1K experiments}
Simulations forced with idealised SST patterns used a 12 month climatology of 
the HadISST data from 1950-2010 as the reference control climate. The division 
between tropics and extra-tropics for these experiments was chosen to be 
$28^{\circ}$N/S, with the tropical forcing applied to the oceans in the zonal 
band bordered by $28^{\circ}$N/S, and the extra-tropical forcing applied from 
$28^{\circ}$N/S to the poles. For the model resolution used this most closely 
divides the oceans in half by area, with slightly more area in the 
extra-tropics.

\subsubsection{Pacemaker experiment}\label{methods:pacemaker}
For the El Ni{\~n}o pattern forcing experiments a canonical El Ni{\~n}o pattern 
was generated using HadISST monthly mean data from a linear regression between 
NINO3 and SSTs, as shown in Figure 1. This pattern was imposed in the tropical 
Pacific between $30^{\circ}$N/S and $155^{\circ}$E to the eastern boundary of 
the Pacific, on to monthly climatological SSTs (i.e. a repeating 12 month 
seasonal cycle).  The values of the anomaly was based on the regression values, 
with a maximum temperature anomaly of 1.41K.  The pattern was oscillated with a 
period of 4 years, peaking in January.  Outside of the tropical Pacific there 
were two scenarios; fixed SSTs using the HadISST 1950-2010 climatology, and the 
slab ocean.  

The slab ocean is essentially a series of 1D ocean columns and assumes a 
constant mixed layer depth of 50 metres. It is forced by flux correction terms 
to have on average the HadISST 1950--2010 SST climatology \citep{Wang2014}. A 
constant mixed layer was chosen in part for simplicity as this model setup is 
intended to be of intermediate complexity, between a simple prescribed SST run 
and a fully coupled dynamic ocean. We use a slab ocean as simple way of 
modelling SST variability, and while we would expect that the response time of 
the remote oceans to an external forcing would scale with mixed layer depth 
(\citealt{Su2005a}, \citealt{Lintner2007}), the constant 50 metre assumption 
gives a first order but realistic response.  


\subsection{One and Two-Dimensional Radiative-Convective Model}
Simplified atmospheric models have long been used to elucidate the important 
mechanisms of otherwise complicated problems (references?). To understand the 
relationship between SSTs, tropospheric temperatures and $T_{land}$ response we 
used a radiative-convective column model. A radiative-convective model is one 
where the atmospheric heating by convection is balanced with radiative cooling 
(ref Manabe, Wetherald 1967 etc) [extra details?] and have been used in a myriad 
of studies, such as the equilibrium response to $CO_2$ (ref) and solar forcings 
(ref) and climate sensitivity (ref), the organisation of deep convection (ref) 
and the atmospheric sctuctural response to SSTs and boundary layer humidity 
(hamish and adam, ref).  There have been studies using RCE over simple land 
surfaces (ref) and more recently with more realistic land representations 
\citep{Rochetin2014}.

We utilised the Massachusetts Institute of Technology (MIT) RCE model. First 
described by \citet{Renno1994} and updated in \citet{Bony2001}.  The convection 
scheme was first described in \citet{Emanuel1991} and improved in 
\citet{Emanuel1999}. Radiative transfer is split into a shortwave scheme  and 
The model can be run as a single column or as a two dimensional model in a zonal 
or meridional directions.  


\begin{itemize}
	\item MIT model, references
\end{itemize}

\paragraph{Model Setups}
\begin{itemize}
	\item Geometry, 1D, 2D
	\item Forcing options; SST, CO2, WTG
	\item Parameter options; land surface, model levels, interactive clouds
\end{itemize}

\paragraph{Experiments}
\begin{itemize}
	\item 1D, WTG setup
	\item 2D, SST forced
	\item 2D, CO2 forced
	\item Parameter experiments
\begin{itemize}
	\item Surface properties
	\item Interactive Clouds
\end{itemize}
\end{itemize}


\subsection{Globally Resolved Energy Balance Model}

\paragraph{Model Description}
\begin{itemize}
	\item GREB model, references
\end{itemize}


\paragraph{Experiments}
\begin{itemize}
	\item AMIP Forcings
	\item Pacemaker forcings
\end{itemize}

%----------------------------------------------------------------------------------------
%	SECTION STATS
%----------------------------------------------------------------------------------------

\section{Statistical methods}

This section will describe the statistical methods applied to the data


\subsection{Regression of land/ocean}\label{ssec:rlo}


All further analysis is based on annual mean anomalies with one exception in 
Section 4.2, which is based on monthly mean anomalies for monthly mean lag-lead
correlations. The land/sea contrast, $R_{L/S}$, is defined by the following 
regression model;

\begin{equation}
T_{land} = R_{L/S} \cdot T_{ocean}
\end{equation}
where
\begin{equation}
	R_{L/S} = \rho_{land,ocean}\cdot \frac{\sigma_{land}}{\sigma_{ocean}}
\end{equation}

With $T_{land}$ and $T_{ocean}$ as the annual mean surface temperature anomalies 
of areal-averaged land and ice-free oceans (global or zonal average), 
respectively, $\rho_{land,ocean}$ is the correlation coefficient between 
$T_{land}$ and $T_{ocean}$, and $\sigma_{land}, \sigma_{ocean}$ are the standard 
deviations of $T_{land}, T_{ocean}$. While regression coefficients are often 
used to measure co-variability, $R_{L/S}$ can also be thought of as an 
amplification factor; how much is the temperature of land amplified compared to 
the SST. A simple ratio, as for the land/sea contrast in global warming, is 
unsuitable as we are dealing with anomalies varying around zero. As a simple 
example, if the anomalous SSTs are +1K the first year and -1K the second, and 
the land temperature is +1.5K then -1.5K respectively, we have a value of 
$R_{L/S} = 1.5$. In this way we consider the regression coefficient an 
appropriate measure of the land/sea temperature contrast in interannual 
variability.
