\chapter{Evidence of the Land/Ocean Contrast in Observations and Models} 

\label{evidence} 

\lhead{Chapter 3. \emph{Evidence of the Land/Ocean Contrast}} 

In this first analysis section we will characterise $R_{L/S}$  in natural 
internal climate variability in observations and model simulations. The focus 
here will be to illustrate that $R_{L/S} > 1.0$  exists on interannual time 
scales in observations and models, but has some significant regional 
differences.

We start the analysis with a look at the observations and the CMIP5 CGCM 
simulations. We then focus on AMIP-type simulations, in which the SST is given 
as the forcing and the $T_{land}$ are responding, which allows us to draw some 
conclusions about the potential of the SST variability as the driving mechanism 
of $T_{land}$ variability. Simplified models will then be discussed.

%----------------------------------------------------------------------------------------
%	Observations
%----------------------------------------------------------------------------------------

\section{Observations}

The land/sea relation of interannual surface temperature variability for 
different regions is shown in Figure \ref{fig:hadcrut} and Table 
\ref{tab:allstats}. Firstly, we can note that in the comparison of the time 
series of the global mean $T_{land}$ and the global mean $T_{ocean}$ they both 
have some common interannual fluctuations (correlation of	0.6; statistically 
significant at the 99\% level), indicating that the global land and ocean have 
co-variability on the interannual time-scales.  The correlation indicates that 
about 1/3 of the total variance of  $T_{land}$ in the global mean is co-variable 
with $T_{ocean}$ and the majority, 2/3, of the total variance of $T_{land}$ is 
independent of $T_{ocean}$, assuming a simple linear relation. We can further 
note that the variability over land is much larger than over  oceans. The ratio 
of the standard deviations is 2.5. The combination of the correlation and the 
ratio in standard deviations leads to the global mean $R_{L/S} =1.43$. Thus the 
variability in surface temperature that is co-variant between the land and the 
oceans is  about 43\% larger in amplitude over land than over oceans.

\begin{figure*}[H]
	\centering
	\includegraphics[width=0.45\textwidth]{{crut4sst.gl.dt}.eps}
	\includegraphics[width=0.45\textwidth]{{crut4sst.tr.dt}.eps}
	\\
	\includegraphics[width=0.45\textwidth]{{crut4sst.nh.dt}.eps}
	\includegraphics[width=0.45\textwidth]{{crut4sst.sh.dt}.eps}
	\caption{Observational annual mean $T_{land}$ and $T_{ocean}$ using detrended 
	HadSST2 and CruTEMP4 data}
	\label{fig:hadcrut}
\end{figure*}

%-----------------------------------
%	Zonal bands
%-----------------------------------

\subsection{Zonal Bands...}

In the next step we look at different zonal bands. We split the globe into a 
tropical band ($30^o$N/S round the equator) and two extra-tropical bands 
(polewards of $30^o$N/S round the equator), with the combined area of the latter 
two bands having the same area as the tropical band. This differs from the 
previous choice of latitude to define the tropics for the experiments with SST 
forcing. In that case $28^o$N/S was chosen in order to allow approximately equal 
ocean areas for the extra-tropical and tropical forcing. First of all it is 
interesting to note that in all three zonal bands $R_{L/S}$ is smaller than in 
the global mean. This suggests that the processes controlling $R_{L/S}$ are more 
effective on global scales or that smaller, regional scale variations tend to 
reduce the value of $R_{L/S}$. In the tropical regions
(Fig.\ref{fig:hadcrut}b) the correlation between $T_{land}$ and $T_{ocean}$ is 
much stronger than for the global means, and although $R_{L/S}=1.2$ is larger 
than unity, it is still smaller than the global value.  Thus the variability in 
surface temperature that is co-variant between the land and the oceans is about 
20\% larger over land than it is over oceans.  The larger correlation also 
indicates that about 2/3 of the total variance of $T_{land}$ in the tropics is 
co-variable with $T_{ocean}$, again assuming a simple linear relation. To some 
extent these differences in the land/sea contrast relative to the global mean 
may reflect the different land and ocean fractions in the tropics. The 
relatively small land fraction suggets that land points are on average closer to 
ocean points and would thus be more strongly linked to the nearby SST  
variability.  However, the differences in the land/sea contrast may also reflect 
differences in physical interactions between land and oceans, which will be 
addressed in the further analysis below.

In the extra-tropical regions of the Northern Hemisphere the land/sea contrast 
is about unity and therefore weaker than in the tropics, but the correlation 
between $T_{land}$ and $T_{ocean}$ is about as large as for the global mean.  
The extra-tropical regions of the Northern Hemisphere are marked by a pronounced 
low-frequency evolution, that is about the same amplitude in both $T_{land}$ and 
$T_{ocean}$. However, some interannual fluctuations appear to be similar in  
$T_{land}$ and $T_{ocean}$ as well (e.g. around the years 1965 and 1990), but 
with much larger amplitudes over land.  In the extra-tropical regions of the 
Southern Hemisphere the land/sea contrast is weaker than in the other zonal 
bands. Again, this may to some extent be related to the distribution of the land 
fraction and in particular to the isolated location of the main southern 
hemispheric land mass of Antarctica.

Since land and ocean areas are unequally distributed over the zonal bands, the 
correlations between the zonal bands may be of interest. In particular, most of 
the interannual SST variability is in the tropical oceans, so one may wonder if 
$T_{land}$ of the extra-tropical regions of the Northern Hemisphere is more 
strongly related to the tropical or global $T_{ocean}$ rather than to the 
extra-tropical Northern Hemisphere $T_{ocean}$. Table \ref{tab:crossrel} shows a 
number of interesting correlations between the zonal bands and between land and 
ocean areas.  First of all we can note that the global mean $T_{ocean}$ is 
strongly dominated by the tropical $T_{ocean}$, which is clearly related to the 
dominant mode of variability ---ENSO--- and to the fact that the tropical oceans 
are the largest part of the global oceans. We can further note that the 
extra-tropical regions of the Northern Hemisphere oceans have a moderate 
positive correlation to the global mean, but not to the tropical $T_{ocean}$.  
The global and Northern Hemisphere $T_{land}$ are nearly identical, as most of 
the land is in the Northern Hemisphere. Although global $T_{ocean}$ is dominated 
by the tropical $T_{ocean}$ the global and the  Northern Hemisphere $T_{land}$ 
have only a moderate correlation to tropical $T_{ocean}$, suggesting only a weak 
direct influence of the tropical $T_{ocean}$ on Northern Hemisphere $T_{land}$.

In summary, in the observations we find a land/sea contrast in the temperature 
variability that has, in most regions, stronger amplitudes over land than over 
oceans. In particular in the tropics there is a strong link between $T_{land}$ 
and $T_{ocean}$ variability, whereas in the extra-tropical regions of the 
Northern Hemisphere the link appears to be much weaker.

\begin{center}
	\begin{table}[h]
		\caption{Annual mean $T_{land}$ and $T_{ocean}$ used to calculate land/sea 
			contrast, ratio of land/sea standard deviations and correlation 
			coefficient between land and sea. Observation data is detrended 
			HadSST2 and CruTEMP4 data. CMIP5 is combined pre-industrial control 
			runs from 35 models, showing one standard deviation between the 
		individual models.  AMIP run was forced with HadISST and detrended.  
	ENSO-like run forced with oscillating canonical ENSO pattern in the tropical 
Pacific, slab ocean elsewhere.}
		\label{tab:allstats}
		\scriptsize
	\begin{tabular}{ l  c  c  c }
		\textit{Data set and region}		& L/S contrast  & L/S correlation & Ratio 
		Std Dev\\ \hline
		\textbf{Observations}\\%\hline
	Global  					& 1.43  & 0.58 & 2.45\\ %\hline
	Tropical  				& 1.19  & 0.81 & 1.48\\ %\hline
	NH Extra-tropics  & 1.00  & 0.61 & 1.64\\ %\hline
	SH Extra-tropics  & 0.22  & 0.10 & 2.28\\ \hline
		\textbf{CMIP5, multi-model mean values}\\%\hline
		Global				& $1.26 \pm 0.23$ & $0.64 \pm 0.13$ & $1.97 \pm 0.26$ \\
		Tropics 			& $1.35 \pm 0.16$ & $0.87 \pm 0.06$ & $1.55 \pm 0.15$ \\ NH 
		extra-tropics & $0.32 \pm 0.30$ & $0.19 \pm 0.18$ & $1.63 \pm 0.29$ \\ SH 
		extra-tropics & $0.03 \pm 0.68$ & $0.01 \pm 0.20$ & $3.41 \pm 0.70$ \\\hline 
		\textbf{AMIP run}\\%\hline
	Global  					& 1.27 & 0.74 & 1.72\\ %\hline
	Tropical  				& 1.26 & 0.88 & 1.43\\ %\hline
	NH Extra-tropics  & 0.53 & 0.34 & 1.59\\ %\hline
	SH Extra-tropics  & 0.64 & 0.16 & 2.41   %\hline
		\\\hline \textbf{ENSO-Slab}\\%\hline
	Global  					& 1.50 & 0.71 & 2.13 \\ %\hline
	Tropical  				& 1.17 & 0.85 & 1.37 \\ %\hline
	NH Extra-tropics  & 0.57 & 0.28 & 2.08 \\ %\hline
	SH Extra-tropics  &-0.42 &-0.11 & 3.69 \\ \hline
	\end{tabular}
	\end{table}
\end{center}

%----------------------------------------------------------------------------------------
%	NCEP
%----------------------------------------------------------------------------------------

\pagebreak

\section{Reanalysis Data (NCEP/NCAR)}

\paragraph{No sig Rl/o, discuss reasons}
\begin{itemize}
	\item Reanalysis looked at in addition to observations.
	\item For tropical, extra-tropical and global regions no significant value 
		of $R_{l/o}$ was found for interannual variability.
	\item Possible reasons; limited data available - temporal and spatial
	\item i.e. Short timeseries. If ENSO events are a major source of 
		variability at a period of about 4 years that only give 5-6 significant 
		events.
	\item Check: global warming in ncep? L/o?
	\item Check: tropical tropospheric response to interannual var.
\end{itemize}


%----------------------------------------------------------------------------------------
%	CMIP5
%----------------------------------------------------------------------------------------

\section{Coupled General Circulation Model Simulations (CMIP5)}

We now explore how CGCM simulations can represent the land/sea contrast in 
natural variability. This helps us to understand the mechanisms behind the 
land/sea contrast as well as providing a much larger data base, which allows us 
to explore the characteristics of the land/sea contrast in more detail. We 
therefore analysise the preindustrial (no external forcings) simulations from 
the CMIP5 data base, using a multi model ensemble of 35 models, with 100 years 
taken from each model. The multi-model ensemble was constructed by combining the 
anomalous surface temperature timeseries from each model to make a single, 3500 
year long timeseries, which was used for all further analysis and statistics. 

In analog to the analysis of the observations (e.g. Fig. 1 and Table 
\ref{tab:allstats}) we summarise the statistics of the land/sea contrast from 
all models for the global, tropical and the two extra-tropical hemispheres in 
Table \ref{tab:allstats} and \ref{tab:crossrel}. The CMIP5 simulations multi 
model mean shows a very similar land/sea contrast in both $R_{L/S}$ and the 
correlation value for both global and tropical means. They also have a very weak 
connection between Southern Hemispheric extra-tropical land and ocean. However, 
in the Northern Hemisphere extra-tropics the models show a weaker link between 
ocean and land variability than observations. The CMIP5 models also do not show 
much impact from the Northern Hemispheric $T_{ocean}$ to global mean 
$T_{ocean}$. In a similar way the Northern Hemispheric $T_{land}$ does not 
dominate global mean $T_{land}$ in the CMIP5 simulation as it does in 
observations.

We can now look at the inter-model variations. The scatter plots in Figure 
\ref{fig:cmip5_scatter} show that amongst the CMIP5 models the land/sea 
correlation in the tropics is linearly related to the global value of the 
land/sea correlation (Fig. \ref{fig:hadcrut}b). Models with a strong land/sea 
connection in the tropics also tend to have a stronger global land/sea 
connection (Fig.  \ref{fig:hadcrut} a).  The extra-tropical interactions are not 
strongly related to the global values. We can further note that the spread in 
the extra-tropical values in both $R_{L/S}$ and the correlation values are much 
larger than in the tropics.  This suggests that the CMIP5 models disagree much 
more on the extra-tropical land/ocean interactions than they do in the tropics. 

The results suggest that tropical values of the land/sea correlation are more 
important in determining the global value, and tropical processes connecting 
ocean and land surface temperatures on these timescales are unrelated to the 
extra-tropics.  Again it should be noted here that global $T_{land}$ is 
dominated by the large land fractions in the extra-tropical Northern Hemisphere.  
The tropical land fraction is much smaller.  In turn, $T_{ocean}$ is dominated 
by the large tropical SST variability. Thus, the strong link between global and 
tropical land/sea correlation suggests that it is the tropical SST variability 
that is a significant cause of the land/sea contrast.

The relationship between global mean $T_{land}$ and the regional SST variability 
is explored next to illustrate which patterns of variability are related to land 
variability. We correlate $T_{land}$ with local SST variability, see Figure 
\ref{fig:cmip_cormap}. Here the timeseries of $T_{land}$ and surface temperature 
anomalies of all CMIP5 models were combined and the annually averaged global 
$T_{land}$ is correlated with surface temperatures. The same analysis was done 
in Dommenget [2009] (Fig. 3a) for observations.  The CMIP5 model results are 
largely similar to the observations as shown in Dommenget [2009], but due to the 
much larger database the emerging pattern is much less noisy and more details 
can be seen.

The CMIP5 models show a strong relationship between global $T_{land}$ and 
tropical ocean and land temperatures. All the tropical land masses are highly 
correlated and there are distinct patterns of high correlations in the tropical 
oceans. There are some similarities in the patterns between the ocean basins; 
there is a minimum at the equator and on the eastern edge of each of the basins. 
Larger correlations in all three tropical ocean basins are on the western side 
of the basin. The highest correlations are in the Indian ocean where there is a 
large region with correlation values above 0.6.  The patterns seem to suggest 
that the SST variability close to the land regions and in the upwind direction 
of the prevailing easterly trade winds are most strongly linked to the global 
$T_{land}$. It is remarkable in this figure that the most dominant pattern of 
SST variability, El Ni{\~n}o, is not directly visible here, as there is a local 
minimum of correlations on the equator.

In the extra-tropical regions we see bands of negative correlations in both 
hemispheres. Thus positive anomalies in the global mean $T_{land}$ are related 
to negative SST anomalies over large parts of the extra-tropical oceans. This 
SST pattern is somewhat similar to the ENSO teleconnections or decadal 
variations of global SST variability (\citealt{Lau1996} \& 
\citealt{Dommenget2008}).  It indicates that changes in the extra-tropical 
atmospheric circulation linked to the tropical SST variability can lead to the 
negative SST correlations in the extra-tropical regions.

To summarise, coupled global climate models are effective in simulating the 
land/sea contrast in natural variability. Tropical values of the land/sea 
correlation and ratio of standard deviations are consistent between models and 
observations. The largest discrepancy between observations and models is in the 
extra-tropics, especially the Northern Hemisphere. These results indicate that 
the physical processes controlling these metrics are well represented by the 
models in the tropics but may not be as well simulated in the extra-tropics.

\begin{center}
	\begin{table}[h]
	\caption{Correlation coefficient of annual mean $T_{land}$ and $T_{ocean}$ 
	between regions, all data detrended}
		\label{tab:crossrel}
	\begin{tabular}{ l  c  c  c}
		\textit{Regions}		&  Observations  & CMIP5 & AMIP \\ \hline
		Global Ocean - Tropical Ocean & 0.81 & 0.92 & 0.90 \\
		Global Ocean - N Hemis ExTr. Ocean & 0.37 & 0.09 & 0.36 \\
		Tropical Ocean - N Hemis ExTr. Ocean & 0.15 & -0.16  & 0.02 \\
		Tropical Ocean- Tropical Land & 0.81 & 0.87 & 0.88 \\
		Tropical Ocean- Global Land & 0.40 & 0.66 & 0.78 \\
		Gobal Land - N Hemis. ExTr. Land & 0.95 & 0.74 & 0.78\\
	\end{tabular}
	\end{table}
\end{center}


%----------------------------------------------------------------------------------------
%	AMIP
%----------------------------------------------------------------------------------------

\section{AMIP simulations}

In the previous section we have characterised the land/sea contrast in 
observations and CGCM simulations. We now address the causality of this link by 
assuming that the land is responding to SST variability. Thus testing the idea 
that the natural SST variability is leading to an amplified response over land.  
We therefore do a series of AMIP-type experiments, in which we prescribe 
historical SST variability globally or in parts of the global oceans and analyse 
the response of the $T_{land}$ and other atmospheric variables.

Figure \ref{fig:amip} shows the same plots as Figure \ref{fig:hadcrut} except 
for an AMIP simulation using the an AGCM forced with the historical global 
HadISST SST variability (see data section for details). The land/sea contrast 
values are largely consistent with observations. The globally averaged values 
are higher than observed; there is a lower ratio of standard deviation between 
land and oceans but a higher correlation. The AMIP tropically averaged values of 
land/sea contrast, correlation and ratio of standard deviations are almost 
identical to observations. AMIP runs are forced only by SSTs, so the high 
correlation between land and ocean surface temperatures in the tropics indicates 
a direct, strong connection from ocean to land. For both the tropical and global 
mean the values of land/sea contrast are larger than unity, indicating that the 
SST forcing is amplified over the continents.

%-----------------------------------
%	SUBSECTION 1
%-----------------------------------

\subsection{Extra-Tropics}

The extra-tropical Northern Hemisphere land/sea contrast value is substantially 
lower than observed. The low values of land/sea contrast in the extra-tropics 
are due to the low correlations between ocean and land; there is still a much 
greater variance of land compared to ocean temperatures. The low correlation of 
annual mean temperature implies that on these timescales the influence of the 
extra-tropical oceans is either less significant or more subtle and less direct 
than in the tropics. If we assume that the models capture the correct ocean-land 
interactions and that the observed extra-tropical land/sea contrast is accurate, 	
then we have to conclude that the extra-tropical land/sea contrast is not forced 
by the SST variability. It may be the atmospheric internal variability forcing 
the extra-tropical SST variability and $T_{land}$, with the $T_{land}$ having 
the larger amplitudes. This picture is consistent with \citet{Barsugli1998}.

In Figure \ref{fig:ftest} f-tests are used to measure the increase in annual 
temperature variability due to SST variability at the surface and at the 300hPa 
pressure level relative to a simulation with fixed SST climatology. Figure  
\ref{fig:ftest} a) and b) show that global SST variability has a substantial 
impact on the tropical atmospheric and surface temperature variability. However, 
in the extra-tropical regions the impact is much weaker, but still statistically 
significant in some regions.

In order to separate the influence of the tropical SST variability from that of 
the extra-tropical SST, we repeat the AMIP experiment forced with the historical 
SST variability just in the tropics or just in the extra-tropical regions. The 
impact of the tropical SST variability is similar to the global SST variability, 
with a clear and strong impact in the tropical regions.  The AMIP simulation 
with just the extra-tropical SST variability has only a very weak to no impact 
on the regional (grid-box scale) atmospheric and surface temperature 
variability.  However, if we compare the global AMIP versus the tropical only 
AMIP run we still can see a somewhat larger increase in variance over land in 
the global AMIP run. This indirectly suggests that the extra-tropical SST 
forcing does play a role, although it is much smaller than the tropical forcing.  
In summary the AMIP experiments suggest a clear tropical SST forcing to the 
atmospheric and land surface temperatures, but a much weaker or no forcing from 
the extra-tropical SST.

It should be noted here, that the AMIP simulations are a good tool to examine 
the observed SST variability, but not necessarily to analyse the CMIP5 SST 
variability, because the structure of the SST patterns may be substantially 
different in the CMIP simulations from those observed. In order to gain a better 
understanding of why the CMIP5 simulations behave differently than the 
observations, it would be useful to do AMIP simulations with the SST variability 
from the different CMIP5 simulations. However, this is beyond the scope of this 
study.

%----------------------------------------------------------------------------------------
%	RCM
%----------------------------------------------------------------------------------------

\section{Radiative Convective Model}

\subsection{Single-Column Model}

\paragraph{WTG Mode, control tropos temp above land leads to $R_{l/o}<1$}
\begin{itemize}
	\item Single column, not usually used for l/s contrast!
	\item WTG mode allows SST forcing, replicate interannual var.
	\item Land responds to SST var with inc. var.
	\item Rl/o depends primarily on evap fraction.
\end{itemize}

\begin{figure}[H]
% 		\includegraphics[width=0.45\textwidth]{{}.eps}
	\caption{Show land/sea contrast in SCM}
\label{fig:SCM_Rlo}
\end{figure}

\subsection{Two-Dimensional Model}

\paragraph{SST forcing, dependance on evap frac, lat}
\begin{itemize}
	\item SST forcing, simulate variability.
	\item Rl/o depends primarily on evap fraction.
	\item Also dependance on mean state temperature and latitude.
\end{itemize}

\begin{figure}[H]
	\includegraphics[width=\textwidth]{{isstcosz}.eps}
	\caption{Land/sea contrast in TCM}
\label{fig:TCM_Rlo}
\end{figure}

%-----------------------------------
%	GREB
%-----------------------------------

\section{GREB Model}

\begin{itemize}
	\item No "variability"
	\item Global warming l/s contrast - ref
	\item What is $R_{l/o}$ for pacemake and amip forcing experients?
\end{itemize}


\processdelayedfloats

